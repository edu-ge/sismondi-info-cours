\documentclass[a4paper,12pt]{article}
\usepackage[T1]{fontenc} 

%%%%%%%%%%%%%%%%%%%%%%%%%%%%%%%% CONSTANTES %%%%%%%%%%%%%%%%%%%%%%%%%%%%%%%%%%%
\newcommand{\numero}{11}                                    %Numéro de la série -1

% Ressources à installer : texlive-science et texlive-lang-french ou texlive-lang-european
%à compiler avec XeLaTeX à cause des image pstricks

\usepackage[utf8]{inputenc}
\usepackage[T1]{fontenc}

\usepackage[french]{babel}
\usepackage{listingsutf8}
\usepackage{tikz,pgf}
\usepackage{amsthm,titlesec}
\usepackage{mdframed} % pour le format des définition, théorèmes...
\usepackage{xcolor,rotating,systeme}
\usepackage[Glenn]{fncychap} %pour le format des chapitres
\usepackage[top=3cm,bottom=3cm,left=3cm,right=2cm,headsep=10pt,a4paper]{geometry} % Page margins
\usepackage{multicol}
\usepackage{enumerate,cancel}
\usepackage{lipsum,minitoc}
%\usepackage{chngcntr}
\usepackage{amsmath,amsfonts,minitoc,amsthm,pgfplots}
\usepackage{mathrsfs}
\usetikzlibrary{arrows, intersections}
\usepackage{mathrsfs}
\usepackage{array,yhmath}
\usepackage{hyperref}
\usepackage{caption}
\usepackage{float}
\usepackage{wrapfig}
\restylefloat{figure}
\usepackage{subfig}

\usepackage{listings}
\usepackage{color}
\lstset{ % General setup for the package
    language=Python,
    basicstyle=\footnotesize,
    basewidth=0.7em,
    numbers=left,
     numberstyle=\tiny,
    frame=lines,
    tabsize=4,
    columns=fixed,
    showstringspaces=false,
    showtabs=false,
    keepspaces,
    commentstyle=\color{gray},
    keywordstyle=\color{blue},
    stringstyle=\color{magenta},
    inputencoding=utf8/latin1,
            extendedchars=true,
            literate=%
            {é}{{\'{e}}}1
            {è}{{\`{e}}}1
            {ê}{{\^{e}}}1
            {ë}{{\¨{e}}}1
            {û}{{\^{u}}}1
            {ù}{{\`{u}}}1
            {â}{{\^{a}}}1
            {à}{{\`{a}}}1
            {î}{{\^{i}}}1
            {ô}{{\^{o}}}1
            {ç}{{\c{c}}}1
            {Ç}{{\c{C}}}1
            {É}{{\'{E}}}1
            {Ê}{{\^{E}}}1
            {À}{{\`{A}}}1
            {Â}{{\^{A}}}1
            {Î}{{\^{I}}}1
    }



\usepackage{lipsum}
\newtheorem{example}{Exemple}

\usepackage{pstricks,pst-plot,pstricks-add}
\usepackage{pst-func,tkz-tab,tkz-euclide}
\usepackage{graphicx}
%macro exo
\newcounter{mtex}[chapter]
\newcommand{\mtexlabel}{\cadrexo{\themtex}}
\newcommand{\cadrexo}[1]{%
\tikz\node[rectangle,minimum size=6mm,rounded corners=2mm,fill=ocre,inner sep=0pt,text width=0.8cm,align=center]{\large\bfseries\textcolor{white}{#1}};}
\definecolor{ocre}{RGB}{196,106,106}
\definecolor{vert}{RGB}{55,120,0}
\newcommand{\mtexlabelpos}[1]{%
\makebox[0pt][r]{\raisebox{#1\baselineskip}[0pt][0pt]{\mtexlabel\quad}}}
\newenvironment{exercice}[1][\empty]{%
\refstepcounter{mtex}%
\trivlist\item\relax%  
\ifx#1\empty\mtexlabelpos{-.7}\else\mtexlabelpos{-.5}%
\hfill\textbf{#1}\hfill\mbox{}\par\fi%
}{\endtrivlist}

\newcommand{\N}{\mathbb{N}}
\newcommand{\Z}{\mathbb Z}
\newcommand{\Q}{\mathbb Q}
\newcommand{\R}{\mathbb R}
\newcommand{\C}{\mathbb C}

\newcommand{\ssi}{\;\;\Leftrightarrow\;\;}

\newcommand{\point}[1]{-- +(#1,#1) -- +(-#1,-#1) -- +(0,0) -- +(-#1,#1) -- +(#1,-#1)} 


\newcommand{\parallele}{\mathbin{\!/\mkern-5mu/\!}} %signe parallèle

\titleformat{\chapter}[frame]
{\setlength\fboxrule{2.25pt}\color{black}}%
{\filleft\scshape\LARGE%
\enspace  Chapitre \thechapter\enspace}%
{20pt}
{\rule{0pt}{30pt}\Huge\scshape\filleft}

\newcommand\fonction[5]{
$\begin{array}{rrll}
#1: & #2 & \rightarrow & #3 \\
  & #4 & \mapsto & #5
\end{array}$
}

%\titleformat
%{\section} % command
%[hang] % shape
%{\bfseries\Large} % format
%{ \thesection\quad} % label
%{0em} % sep
%{
%} % before-code
%[
%\vspace{-1ex}%
%\textcolor{blue!60}{\rule{10cm}{3pt}}
%] % after-code


%\theoremstyle{break}
\newtheoremstyle{definition-style}  %style des theoreme
{}               
{}               
{}                   
{}                   
{\bf\sffamily} 
{~:\\[0.3cm]}                 
{.5em}               
{\thmname{#1}\thmnumber{ #2}\thmnote{ (#3)}}

\mdfdefinestyle{defi-frame}{
linewidth=10pt, %
linecolor= blue!50, % 
backgroundcolor= blue!20,
topline=false, %
bottomline=false, %
rightline=false,%
leftmargin=0pt, %
innerleftmargin=15pt, %
innerrightmargin=1em, 
rightmargin=0pt, % 
innertopmargin=-2pt,%
innerbottommargin=6pt, % 
splittopskip=\topskip, %
%splitbottomskip=\topskip, %
}% 

\mdfdefinestyle{methode-frame}{
	linewidth=10pt, %
linecolor= gray!70, % 
backgroundcolor= white,
topline=false, %
bottomline=false,%
rightline=false,%
leftmargin=0pt, %
innerleftmargin=15pt, %
innerrightmargin=1em, 
rightmargin=0pt, % 
innertopmargin=-2pt,%
innerbottommargin=6pt, % 
splittopskip=\topskip, %
%splitbottomskip=\topskip, %
}% 


\mdfdefinestyle{thm-frame}{
linewidth=10pt, %
linecolor= red!50, % 
backgroundcolor= orange!30,
topline=false, %
bottomline=false, %
rightline=false,%
leftmargin=0pt, %
innerleftmargin=15pt, %
innerrightmargin=1em, 
rightmargin=0pt, % 
innertopmargin=-2pt,%
innerbottommargin=6pt, % 
splittopskip=\topskip, %
%splitbottomskip=\topskip, %
}% 

\mdfdefinestyle{lem-frame}{
linewidth=10pt, %
linecolor= yellow!50, % 
backgroundcolor= yellow!10,
topline=false, %
bottomline=false, %
rightline=false,%
leftmargin=0pt, %
innerleftmargin=15pt, %
innerrightmargin=1em, 
rightmargin=0pt, % 
innertopmargin=-2pt,%
innerbottommargin=6pt, % 
splittopskip=\topskip, %
%splitbottomskip=\topskip, %
}% 

\mdfdefinestyle{notation-frame}{
linewidth=10pt, %
linecolor= blue!30, % 
backgroundcolor= blue!10,
topline=false, %
bottomline=false, %
rightline=false,%
leftmargin=0pt, %
innerleftmargin=15pt, %
innerrightmargin=1em, 
rightmargin=0pt, % 
innertopmargin=-2pt,%
innerbottommargin=6pt, % 
splittopskip=\topskip, %
%splitbottomskip=\topskip, %
}% 

\mdfdefinestyle{remarque-frame}{
linewidth=0pt, %
linecolor= black!0, % 
backgroundcolor= blue!0,
topline=false, %
bottomline=false, %
rightline=false,%
leftmargin=0pt, %
innerleftmargin=25pt, %
innerrightmargin=25pt, 
rightmargin=0pt, % 
innertopmargin=-2pt,%
innerbottommargin=6pt, % 
splittopskip=\topskip, %
%splitbottomskip=\topskip, %
}% 
\surroundwithmdframed[style=defi-frame]{defi}

\surroundwithmdframed[style=thm-frame]{thm}

\surroundwithmdframed[style=lem-frame]{lem}

\surroundwithmdframed[style=notation-frame]{notation}

\surroundwithmdframed[style=remarque-frame]{remarque}

\surroundwithmdframed[style=remarque-frame]{remarques}

\surroundwithmdframed[style=methode-frame]{methode}

\theoremstyle{definition-style}

%pour avoir un compteur commun a remarque et remarques
\newcounter{compteurremarque}[chapter]
\renewcommand{\thecompteurremarque}{\arabic{compteurremarque}} %définition de l'affichage du compteur

\newtheorem{defi}{Définition}[chapter]
\renewcommand{\thedefi}{\arabic{defi}} %pour ne pas avoir le numéro du chapitre dans la numération de la def
\newtheorem{thm}{Théorème}[chapter]
\renewcommand{\thethm}{\arabic{thm}}
\newtheorem{lem}{Lemme}[chapter]
\renewcommand{\thelem}{\arabic{lem}}
\newtheorem{notation}{Notation}[chapter]
\renewcommand{\thenotation}{\arabic{notation}}
%\newtheorem{remarque}{Remarque}[chapter]
%\renewcommand{\theremarque}{\arabic{remarque}}
%\newtheorem{remarques}{Remarques}[chapter]
%\renewcommand{\theremarques}{\arabic{remarques}}
\newtheorem{methode}{Méthode}[chapter]
\renewcommand{\themethode}{\arabic{methode}}


\newtheorem{remarque}[compteurremarque]{Remarque}
\newtheorem{remarques}[compteurremarque]{Remarques} %remarque et remarques prennent compteurremarque comme compteur commun

\lstMakeShortInline[columns=fixed]|

%%%%%%%% Commandes de Matthieu

%-------------------------------------------------------------------------------
%---- Eclairage : en encadré sur fond jaune avec symbôle "ampoule" à gauche ----
%-------------------------------------------------------------------------------
\definecolor{coleclairage}{RGB}{255 , 221 , 156}
\definecolor{contoureclairage}{RGB}{255 , 192 , 0}
\newenvironment{eclairage}
{
	\begin{center}%
		\begin{tikzpicture}%
			\node[rectangle, draw=contoureclairage, top color=coleclairage!50, bottom color=coleclairage!140, rounded corners=5pt, inner xsep=5pt, inner ysep=6pt, outer ysep=10pt]\bgroup                     
			\begin{minipage}{0.98\linewidth}
				\begin{minipage}{0.08\linewidth}\centerline{\includegraphics[scale=1]{images/Symbole_eclairage.png}}\end{minipage}
				\begin{minipage}{0.89\linewidth}\itshape\footnotesize
				}
				{                		
				\end{minipage}
			\end{minipage}\egroup;%
		\end{tikzpicture}%
	\end{center}%
}



\newcommand{\putfigure}[6]{
    \begin{minipage}{#1\textwidth}
    \centering
        \includegraphics[width=#2\textwidth,height=#3\textheight]{#4}
        \captionof{figure}{#5}
        \label{#6}
    \end{minipage}
}


\newcommand{\itemb}[1]{\item \textbf{#1}}



%-------------------------------------------------------------------------------
%---- apprendre : en encadré sur fond jaune avec symbole "ampoule" à gauche ----
%-------------------------------------------------------------------------------
\definecolor{colapprendre}{RGB}{50,205,50}
\definecolor{contourapprendre}{RGB}{34,139,34}
\newenvironment{apprendre}
{
	\begin{center}%
		\begin{tikzpicture}%
			\node[rectangle, draw=contourapprendre, top color=colapprendre!10, bottom color=colapprendre!50, rounded corners=5pt, inner xsep=5pt, inner ysep=6pt, outer ysep=10pt]\bgroup                     
			\begin{minipage}{0.98\linewidth}
				\begin{minipage}{0.08\linewidth}\centerline{\includegraphics[width=30px]{images/Symbole_learn.png}}\end{minipage}
				\begin{minipage}{0.89\linewidth}\itshape\footnotesize
				}
				{                		
				\end{minipage}
			\end{minipage}\egroup;%
		\end{tikzpicture}%
	\end{center}%
}

\definecolor{colimportant}{RGB}{247 , 189 , 164}
\definecolor{contourimportant}{RGB}{237 , 125 , 49}
\newenvironment{important}
{
	\begin{center}%
		\begin{tikzpicture}%
			\node[rectangle, draw=contourimportant, top color=colimportant!50, bottom color=colimportant!140, rounded corners=5pt, inner xsep=5pt, inner ysep=6pt, outer ysep=10pt]\bgroup                     
			\begin{minipage}{0.08\linewidth}\centerline{\includegraphics[scale=0.8]{images/Symbole_attention.png}}\end{minipage}
			\begin{minipage}{0.89\linewidth}
			}
			{                		
			\end{minipage}\egroup;
		\end{tikzpicture}%
	\end{center}%
}

%----------------------------------------------
\newcounter{compteurmonprogramme}
\setcounter{compteurmonprogramme}{1}
\newenvironment{monprogramme}
{
	\begin{center}%
		\begin{tikzpicture}%
			\node[rectangle, draw=black, rounded corners=5pt, inner xsep=5pt, inner ysep=6pt, outer ysep=10pt]\bgroup                     
			\begin{minipage}{0.98\linewidth}
				{\bf Programme \thechapter.\thecompteurmonprogramme\;:}
				
				\vspace{2mm}\hspace{7.5mm}
				\begin{minipage}{0.93\linewidth}
				}
				{                		
				\end{minipage}
				\stepcounter{compteurmonprogramme}
			\end{minipage}\egroup;%
		\end{tikzpicture}%
	\end{center}%
}



%------------------------------------------------
%---- Définition : en encadré sur fond blanc ----
%------------------------------------------------
\newcounter{compteurdef}
\setcounter{compteurdef}{1}
\newenvironment{mydefinition}
{
	\begin{center}%
	\begin{tikzpicture}%
		\node[rectangle, draw=black, rounded corners=5pt, inner xsep=5pt, inner ysep=6pt, outer ysep=10pt]\bgroup                    
		\begin{minipage}{0.98\linewidth}
			{\bf {\underline {Définition \thechapter.\thecompteurdef}}}
			
			\vspace{2mm}\hspace{4.5mm}
			\begin{minipage}{0.93\linewidth}
			}
			{                		
			\end{minipage}
			\stepcounter{compteurdef}
		\end{minipage}\egroup;%
	\end{tikzpicture}%
\end{center}%
}

\newenvironment{mydefinitions}
{
	\begin{center}%
		\begin{tikzpicture}%
			\node[rectangle, draw=black, rounded corners=5pt, inner xsep=5pt, inner ysep=6pt, outer ysep=10pt]\bgroup                    
			\begin{minipage}{0.98\linewidth}
				{\bf {\underline {Définitions \thechapter.\thecompteurdef}}}
				
				\vspace{3mm}\hspace{4.5mm}
				\begin{minipage}{0.93\linewidth}\slshape
					\begin{itemize}\setlength{\itemsep}{1mm}
					}
					{                		
				\end{itemize}\end{minipage}
				\stepcounter{compteurdef}
			\end{minipage}\egroup;%
		\end{tikzpicture}%
	\end{center}%
}



%------------------------------------------------
%---- Exemple : en encadré sur fond blanc ----
%------------------------------------------------
\newcounter{compteurex}
\setcounter{compteurex}{1}
\newenvironment{myexample}{
	\begin{center}
	\vspace{-3mm}
	\begin{minipage}{1\linewidth}
		\vspace{2mm}
		{\textsl {\underline {Exemple \thechapter.\thecompteurex}}}
		
		\vspace{2mm}\hspace{2.5mm}
		\begin{minipage}{1\linewidth}
			\begin{mdframed}[topline=false,rightline=false,bottomline=false]
}
{
			\end{mdframed}
		\end{minipage}
		\stepcounter{compteurex}
	\end{minipage}
	\end{center}
}
\newenvironment{myexamples}{
	\begin{center}
		\vspace{-3mm}
		\begin{minipage}{1\linewidth}
			\vspace{2mm}
			{\textsl {\underline {Exemples \thechapter.\thecompteurex}}}
			
			\vspace{2mm}\hspace{2.5mm}
			\begin{minipage}{1\linewidth}%\slshape
				\begin{mdframed}[topline=false,rightline=false,bottomline=false]
					\begin{enumerate}\setlength{\itemsep}{1mm}
				}
				{
					\end{enumerate}
				\end{mdframed}
			\end{minipage}
			\stepcounter{compteurex}
		\end{minipage}
	\end{center}
}



\newtheorem*{question}{Question}



%%%%%%%%%%%%%%%%%%%%%%%%%%%%%%%%%%%%%%%%%%%%%%%%%%%%%%%%%%%%%%%%%%%%%%%%%%%%
%%%%%%%%%%%%%%%%%%%%%%%%%%%   LABELS activite   %%%%%%%%%%%%%%%%%%%%%%%%%%%
%%%%%%%%%%%%%%%%%%%%%%%%%%%%%%%%%%%%%%%%%%%%%%%%%%%%%%%%%%%%%%%%%%%%%%%%%%%%
\newcommand{\act}{\textbf{\textsl{Activité \arabic{compteuract}}} \vspace*{1mm}\\ \addtocounter{compteuract}{1}}
\newcommand{\actnomme}[1]{{\bf Activité \arabic{compteuract} {\textsl{\small (#1)}}}\vspace*{1mm}\\  \addtocounter{compteuract}{1}}
\newcounter{compteuract}
\setcounter{compteuract}{1}
\newcommand{\getactcompteur}{{\the\numexpr \arabic{compteuract} - 1 \relax}}



%%%%%%%%%%%%%%%%%%%%%%%%%%%%%%%%%%%%%%%%%%%%%%%%%%%%%%%%%%%%%%%%%%%%%%%%%%%%
%%%%%%%%%%%%%%%%%%%%%%%%%%%   LABELS EXERCICES   %%%%%%%%%%%%%%%%%%%%%%%%%%%
%%%%%%%%%%%%%%%%%%%%%%%%%%%%%%%%%%%%%%%%%%%%%%%%%%%%%%%%%%%%%%%%%%%%%%%%%%%%
\newcommand{\exo}{\textbf{\textsl{Exercice \arabic{compteurexo}}} \vspace*{1mm}\\ \addtocounter{compteurexo}{1}}
\newcommand{\exonomme}[1]{{\bf Exercice \arabic{compteurexo} {\textsl{\small (#1)}}}\vspace*{1mm}\\  \addtocounter{compteurexo}{1}}
\newcommand{\eexo}{\vspace{5mm}} % espace pour séparer les exercices
\newcounter{compteurexo}
\setcounter{compteurexo}{1}
\newcommand{\getexocompteur}{{\the\numexpr \arabic{compteurexo} - 1 \relax}}
\definecolor{light-gray}{gray}{0.95} 
\lstset{language=Python, upquote=true, showstringspaces=false, columns=fullflexible, basicstyle = \ttfamily, backgroundcolor = \color{light-gray}, numbers=right, stepnumber=1, keywordstyle=\color{blue}, stringstyle=\color{magenta}, commentstyle=\color{red}, breaklines=true,} 

\begin{document}
%		\title{\vspace{-3cm}Série 1}
%		\date{\vspace{-2cm}}
%		\maketitle

\fancyhead[CO]{\sc Série \arabic{section} \hspace{0.5mm}}
%\fancyhead[CE]{\sc Série \arabic{section} \hspace{0.5mm}}
\setcounter{section}{\numero}

\section{Structure de contrôle et utilisation de fonctions}
\Opensolutionfile{mycor}[cor01]
%%%%%%%%%%%%%%%%%%%%%%%%%%%%%%%% EXERCICE %%%%%%%%%%%%%%%%%%%%%%%%%%%%
\exo{}[Compréhension de code]  ~\\ 
\begin{lstlisting}
n = 4
for i in range(n):
    print(i)
\end{lstlisting}
Que va afficher ce code Python ?
\begin{lstlisting}https://www.overleaf.com/project/614b7a33668562d2ad26e0cd
n = 4
for i in range(2, n):
    print(i)
\end{lstlisting}
Et celui-ci ?
\begin{correction}
		~\\ \vspace{-5pt}
		Le premier exemple affichera successivement les valeurs 1, 2 et 3.
		
		Le second affichera uniquement 2 et 3.
	\end{correction}
\finexo
%%%%%%%%%%%%%%%%%%%%%%%%%%%%%%%% EXERCICE %%%%%%%%%%%%%%%%%%%%%%%%%%%%
\exo{}[Compréhension de code]  ~\\ 
Voici un code en Python:
\begin{lstlisting}
x = 1
for i in range(4):
    x = x + i
\end{lstlisting}
Quelle est la valeur finale de x ?
\begin{correction}
		~\\ \vspace{-5pt}
		%\\lstinputlisting[numbers=none]{codes/corr_exo_verif_age.py}
		La variable i prend successivement les valeurs 0, 1, 2 et 3. Donc x prend les valeurs 1, 2, 4 et 7.
\end{correction}
\finexo
%%%%%%%%%%%%%%%%%%%%%%%%%%%%%%%% EXERCICE %%%%%%%%%%%%%%%%%%%%%%%%%%%%
\exo{}[Compréhension de code]  ~\\ 
Voici une fonction définie en Python:
\begin{lstlisting}
def f(x):
    for d in range(2, x):
        if x % d == 0:
            return d
\end{lstlisting}
Que renvoie la fonction f si le paramètre x a la valeur 15 ?
\begin{correction}
		~\\ \vspace{-5pt}
		%\\lstinputlisting[numbers=none]{codes/corr_exo_verif_age.py}
		Le nombre 15 est divisible par 3, donc la fonction renvoie 3. Ce retour interrompt le code.
\end{correction}
\finexo
%%%%%%%%%%%%%%%%%%%%%%%%%%%%%%%% EXERCICE %%%%%%%%%%%%%%%%%%%%%%%%%%%%
\exo{}[Compréhension de code]  ~\\ 
Que va afficher le code suivant ?
\begin{lstlisting}
compteur = 0
while compteur < 12:
    if(compteur%3)==0:
        print("Zig")
    elif(compteur%3)==1:
        print("Zag")
    elif(compteur%3)==2:
        print("Zoug")
    compteur += 1
\end{lstlisting}
\begin{correction}
Le résultat sera affiché sera:
\begin{lstlisting}
Zig
Zag
Zoug
Zig
Zag
Zoug
Zig
Zag
Zoug
Zig
Zag
Zoug
\end{lstlisting}
\end{correction}
\finexo
%%%%%%%%%%%%%%%%%%%%%%%%%%%%%%%% EXERCICE %%%%%%%%%%%%%%%%%%%%%%%%%%%%
\exo{}[Compréhension de code]  ~\\ 
Que va afficher le code suivant si vous vous prénommez Cindy ?
\begin{lstlisting}
texte_debut = "Bonjour"
texte_fin = "Au revoir"
prenom = input("Quel est votre prénom ? : ")
for i in range(4):
    print(texte_debut, prenom + ".")
    if i == 3 :
        print(texte_fin, prenom + ".")
\end{lstlisting}
\begin{correction}
Le résultat sera affiché sera:
\begin{lstlisting}
Bonjour Cindy.
Bonjour Cindy.
Bonjour Cindy.
Bonjour Cindy.
Au revoir Cindy.
\end{lstlisting}
	\end{correction}

\finexo
%%%%%%%%%%%%%%%%%%%%%%%%%%%%%%%% EXERCICE %%%%%%%%%%%%%%%%%%%%%%%%%%%%
\exo{}[Compréhension de code]  ~\\ 
Après le code Python qui suit, quelles sont les valeurs finales de x et de y ?
\begin{lstlisting}
x = 4
while x > 0:
    y = 0
    while y < x:
        y = y + 1
        x = x - 1
\end{lstlisting}
\begin{correction}
		~\\ \vspace{0pt}
		%\\lstinputlisting[numbers=none]{codes/corr_exo_verif_age.py}
Les valeurs de x sont strictement décroissantes et la valeur de y est remise à 0 dès qu'elle n'est plus strictement inférieure à celle de x. Au dernier passage dans la boucle interne, y vaut 0, et x vaut 1 : y prend alors la valeur de 1 et x la valeur de 0;
on sort de la boucle interne puis de la boucle externe.
On a une boucle infinie si on remplace y=0 par y=1.
\end{correction}

\finexo
%%%%%%%%%%%%%%%%%%%%%%%%%%%%%%%% EXERCICE %%%%%%%%%%%%%%%%%%%%%%%%%%%%
\exo{}[Compréhension de code]  ~\\ 
Analysez le script ci-dessous et indiquez ce qui sera affiché.
\begin{lstlisting}
# definition de la fonction
def table_7():
    for n in range(1, 13):
        print(n*7, end=' ')
    
# appel de la fonction
table_7()
\end{lstlisting}
\begin{correction}
		~\\ \vspace{-5pt}
		%\\lstinputlisting[numbers=none]{codes/corr_exo_verif_age.py}
La variable n prend successivement les valeurs de 1 à 12.
Le résultat sera affiché sera:
\begin{lstlisting}
7 14 21 28 35 42 49 56 63 70 77 84 
\end{lstlisting}
end=' '  dans la fonction print indique de ne pas aller à la ligne et de séparer chaque print(n*7) par un espace.
	\end{correction}

\finexo
%%%%%%%%%%%%%%%%%%%%%%%%%%%%%%%% EXERCICE %%%%%%%%%%%%%%%%%%%%%%%%%%%%
\exo{}[Erreur de syntaxe]  ~\\ 
Chacun des scripts ci-dessous contient une erreur de syntaxe différente. Laquelle ? Après avoir deviné, vous pouvez tester en mode interactif : soyez attentifs au message d’erreur affiché !
\begin{lstlisting}
def table_7()
    for n in range(13):
        print(n*7,end=' ')
\end{lstlisting} 
\begin{lstlisting}           
def table_7:
    for n in range(13):
        print(n*7,end=' ')        
\end{lstlisting} 
\begin{lstlisting}   
table_7():
    for n in range(13):
        print(n*7,end=' ')
\end{lstlisting} 
\begin{lstlisting}       
def table_7():
for n in range(13):
    print(n*7,end=' ')
\end{lstlisting}
\begin{correction}
	~\\ \vspace{-5pt}
	%\\lstinputlisting[numbers=none]{codes/corr_exo_verif_age.py}
	\begin{enumerate}
        \item Il manque \lstinline{:} après \lstinline{def table_7()}
        \item Il manque \lstinline{()} entre \lstinline{table_7 et :}
        \item Il manque \lstinline{def} au début de la fonction \lstinline{table_7():}
        \item la boucle \lstinline{for} n'est pas indentée correctement dans la fonction.
    \end{enumerate}
\end{correction}
\finexo
\newpage
%%%%%%%%%%%%%%%%%%%%%%%%%%%%%%%% EXERCICE %%%%%%%%%%%%%%%%%%%%%%%%%%%%
\exo{}[Les différences]  ~\\ 
Analysez les deux scripts suivants : quelles sont les différences ? Constatez-vous une ou plusieurs erreurs de programmation ?
\begin{lstlisting}
def est_pair(n):
    if n % 2 == 0:
        print('True')
    else:
        print('False')

nombre = 14
if not est_pair(nombre):
    print(nombre,' est impair')
else:
    print(nombre,' est pair')
\end{lstlisting} 
\begin{lstlisting}  
def est_pair(n):
    if n % 2 == 0:
        return True
    else:
        return False

nombre = 14
if not est_pair(nombre):
    print(nombre,' est impair')
else:
    print(nombre,' est pair')
\end{lstlisting}
Est-ce que le code de la 2ème fonction peut être simplifiée ? Si oui comment ?
\begin{correction}
	~\\ \vspace{-5pt}
	%\\lstinputlisting[numbers=none]{codes/corr_exo_verif_age.py}
	\begin{enumerate}
        \item Dans le 1er code, la fonction \lstinline{est_pair} ne retourne pas de valeur codée. La fonction retourne par défaut \lstinline{None}. Le programme affichera \lstinline{True} ensuite \lstinline{14 est impair} ce qui est faux !
        \item Le 2ème code est correct.
        \item Oui, la fonction peut être simplifiée.
    \end{enumerate}
\begin{lstlisting}  
def est_pair(n):
    if n % 2 == 0:
        return True
    return False
\end{lstlisting}
Il n'est pas nécessaire d'ajouter le bloc \lstinline{else:} car l'instruction \lstinline{return True} interrompt l’exécution de la fonction ! Mais il est encore possible de simplifier en retournant directement l'évaluation du test \lstinline{n % 2 == 0} qui est soit True, soit False. Notez que les parenthèses ne sont pas nécessaire pour encadrer l'expression évaluée.
\begin{lstlisting}  
def est_pair(n):
    return n % 2 == 0
\end{lstlisting}

\end{correction}

\finexo

%%%%%%%%%%%%%%%%%%%%%%%%%%%%%%%% EXERCICE %%%%%%%%%%%%%%%%%%%%%%%%%%%%
\exo{}[Les paramètres]  ~\\ 
On peut aussi passer à la fonction une liste de {\it paramètres}. Analysez le script ci-dessous et indiquez son résultat.
\begin{lstlisting}
# definition de la fonction    
def tableMulti(base, debut, fin):
    for n in range(debut, fin+1):
        print(n*base, end=' ')

# appel de la fonction
tableMulti(8, 13, 17)
\end{lstlisting}
\begin{correction}
	~\\ \vspace{-5pt}
	%\\lstinputlisting[numbers=none]{codes/corr_exo_verif_age.py}
    Les nombres suivants seront affichés:
    \lstinline{104 112 120 128 136}
\end{correction}
\finexo

\newpage
%%%%%%%%%%%%%%%%%%%%%%%%%%%%%%%% EXERCICE %%%%%%%%%%%%%%%%%%%%%%%%%%%%
\exo{}[Variables locales - variables globales]  ~\\ 
En fonction du point d'exécution d'un programme, la visibilité des variables n'est pas la même: les variables définies dans le contexte de base sont dites globales, les variables définies dans une fonction sont locales.

Donner le résultat ou l'erreur affiché pour les deux codes suivants:
\begin{lstlisting}
a = 1
b = 2
def test(a,b):
    a = 5
    print(a,b)

test(3,4)
print(a,b)
\end{lstlisting} 

\begin{lstlisting}
a = 1
b = 2
def test(a,b):
    print(a,b)
    
def test2(c,d):
    print(a,b)
    
def test3():
    c = 3
    d = 4

test(a,b)
test2(3,4)
test3()
print(c,d)
\end{lstlisting}
\begin{correction}
	~\\ \vspace{-5pt}
	%\\lstinputlisting[numbers=none]{codes/corr_exo_verif_age.py}
    Premier programme, les nombres suivants seront affichés:
   \begin{lstlisting}
   5 4
   1 2
   \end{lstlisting}
    Deuxième programme, les nombres suivants seront affichés:
   \begin{lstlisting}
   1 2
   1 2
    ... Erreur: c n'est pas définie...
       print(c,d)
   NameError: name 'c' is not defined
   \end{lstlisting}

\end{correction}


\finexo

% Solution		
		\newpage
		\setcounter{page}{1}
		\setcounter{section}{\numero}
		\Closesolutionfile{mycor}
		\titleformat{\section}[hang]{\Large \bfseries}{Corrigé Série \thesection:\ }{0pt}{}
		

		\fancyhead[CO]{\sc Corrigé Série \arabic{section} \hspace{0.5mm}}
		\section{}
		\Readsolutionfile{mycor}
	\end{document}