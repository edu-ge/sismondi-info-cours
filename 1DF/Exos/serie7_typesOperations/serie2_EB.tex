\documentclass[a4paper,12pt]{article} 

%%%%%%%%%%%%%%%%%%%%%%%%%%%%%%%% CONSTANTES %%%%%%%%%%%%%%%%%%%%%%%%%%%%%%%%%%%
\newcommand{\numero}{1}                                    %Numéro de la série -1

% Ressources à installer : texlive-science et texlive-lang-french ou texlive-lang-european
%à compiler avec XeLaTeX à cause des image pstricks

\usepackage[utf8]{inputenc}
\usepackage[T1]{fontenc}

\usepackage[french]{babel}
\usepackage{listingsutf8}
\usepackage{tikz,pgf}
\usepackage{amsthm,titlesec}
\usepackage{mdframed} % pour le format des définition, théorèmes...
\usepackage{xcolor,rotating,systeme}
\usepackage[Glenn]{fncychap} %pour le format des chapitres
\usepackage[top=3cm,bottom=3cm,left=3cm,right=2cm,headsep=10pt,a4paper]{geometry} % Page margins
\usepackage{multicol}
\usepackage{enumerate,cancel}
\usepackage{lipsum,minitoc}
%\usepackage{chngcntr}
\usepackage{amsmath,amsfonts,minitoc,amsthm,pgfplots}
\usepackage{mathrsfs}
\usetikzlibrary{arrows, intersections}
\usepackage{mathrsfs}
\usepackage{array,yhmath}
\usepackage{hyperref}
\usepackage{caption}
\usepackage{float}
\usepackage{wrapfig}
\restylefloat{figure}
\usepackage{subfig}

\usepackage{listings}
\usepackage{color}
\lstset{ % General setup for the package
    language=Python,
    basicstyle=\footnotesize,
    basewidth=0.7em,
    numbers=left,
     numberstyle=\tiny,
    frame=lines,
    tabsize=4,
    columns=fixed,
    showstringspaces=false,
    showtabs=false,
    keepspaces,
    commentstyle=\color{gray},
    keywordstyle=\color{blue},
    stringstyle=\color{magenta},
    inputencoding=utf8/latin1,
            extendedchars=true,
            literate=%
            {é}{{\'{e}}}1
            {è}{{\`{e}}}1
            {ê}{{\^{e}}}1
            {ë}{{\¨{e}}}1
            {û}{{\^{u}}}1
            {ù}{{\`{u}}}1
            {â}{{\^{a}}}1
            {à}{{\`{a}}}1
            {î}{{\^{i}}}1
            {ô}{{\^{o}}}1
            {ç}{{\c{c}}}1
            {Ç}{{\c{C}}}1
            {É}{{\'{E}}}1
            {Ê}{{\^{E}}}1
            {À}{{\`{A}}}1
            {Â}{{\^{A}}}1
            {Î}{{\^{I}}}1
    }



\usepackage{lipsum}
\newtheorem{example}{Exemple}

\usepackage{pstricks,pst-plot,pstricks-add}
\usepackage{pst-func,tkz-tab,tkz-euclide}
\usepackage{graphicx}
%macro exo
\newcounter{mtex}[chapter]
\newcommand{\mtexlabel}{\cadrexo{\themtex}}
\newcommand{\cadrexo}[1]{%
\tikz\node[rectangle,minimum size=6mm,rounded corners=2mm,fill=ocre,inner sep=0pt,text width=0.8cm,align=center]{\large\bfseries\textcolor{white}{#1}};}
\definecolor{ocre}{RGB}{196,106,106}
\definecolor{vert}{RGB}{55,120,0}
\newcommand{\mtexlabelpos}[1]{%
\makebox[0pt][r]{\raisebox{#1\baselineskip}[0pt][0pt]{\mtexlabel\quad}}}
\newenvironment{exercice}[1][\empty]{%
\refstepcounter{mtex}%
\trivlist\item\relax%  
\ifx#1\empty\mtexlabelpos{-.7}\else\mtexlabelpos{-.5}%
\hfill\textbf{#1}\hfill\mbox{}\par\fi%
}{\endtrivlist}

\newcommand{\N}{\mathbb{N}}
\newcommand{\Z}{\mathbb Z}
\newcommand{\Q}{\mathbb Q}
\newcommand{\R}{\mathbb R}
\newcommand{\C}{\mathbb C}

\newcommand{\ssi}{\;\;\Leftrightarrow\;\;}

\newcommand{\point}[1]{-- +(#1,#1) -- +(-#1,-#1) -- +(0,0) -- +(-#1,#1) -- +(#1,-#1)} 


\newcommand{\parallele}{\mathbin{\!/\mkern-5mu/\!}} %signe parallèle

\titleformat{\chapter}[frame]
{\setlength\fboxrule{2.25pt}\color{black}}%
{\filleft\scshape\LARGE%
\enspace  Chapitre \thechapter\enspace}%
{20pt}
{\rule{0pt}{30pt}\Huge\scshape\filleft}

\newcommand\fonction[5]{
$\begin{array}{rrll}
#1: & #2 & \rightarrow & #3 \\
  & #4 & \mapsto & #5
\end{array}$
}

%\titleformat
%{\section} % command
%[hang] % shape
%{\bfseries\Large} % format
%{ \thesection\quad} % label
%{0em} % sep
%{
%} % before-code
%[
%\vspace{-1ex}%
%\textcolor{blue!60}{\rule{10cm}{3pt}}
%] % after-code


%\theoremstyle{break}
\newtheoremstyle{definition-style}  %style des theoreme
{}               
{}               
{}                   
{}                   
{\bf\sffamily} 
{~:\\[0.3cm]}                 
{.5em}               
{\thmname{#1}\thmnumber{ #2}\thmnote{ (#3)}}

\mdfdefinestyle{defi-frame}{
linewidth=10pt, %
linecolor= blue!50, % 
backgroundcolor= blue!20,
topline=false, %
bottomline=false, %
rightline=false,%
leftmargin=0pt, %
innerleftmargin=15pt, %
innerrightmargin=1em, 
rightmargin=0pt, % 
innertopmargin=-2pt,%
innerbottommargin=6pt, % 
splittopskip=\topskip, %
%splitbottomskip=\topskip, %
}% 

\mdfdefinestyle{methode-frame}{
	linewidth=10pt, %
linecolor= gray!70, % 
backgroundcolor= white,
topline=false, %
bottomline=false,%
rightline=false,%
leftmargin=0pt, %
innerleftmargin=15pt, %
innerrightmargin=1em, 
rightmargin=0pt, % 
innertopmargin=-2pt,%
innerbottommargin=6pt, % 
splittopskip=\topskip, %
%splitbottomskip=\topskip, %
}% 


\mdfdefinestyle{thm-frame}{
linewidth=10pt, %
linecolor= red!50, % 
backgroundcolor= orange!30,
topline=false, %
bottomline=false, %
rightline=false,%
leftmargin=0pt, %
innerleftmargin=15pt, %
innerrightmargin=1em, 
rightmargin=0pt, % 
innertopmargin=-2pt,%
innerbottommargin=6pt, % 
splittopskip=\topskip, %
%splitbottomskip=\topskip, %
}% 

\mdfdefinestyle{lem-frame}{
linewidth=10pt, %
linecolor= yellow!50, % 
backgroundcolor= yellow!10,
topline=false, %
bottomline=false, %
rightline=false,%
leftmargin=0pt, %
innerleftmargin=15pt, %
innerrightmargin=1em, 
rightmargin=0pt, % 
innertopmargin=-2pt,%
innerbottommargin=6pt, % 
splittopskip=\topskip, %
%splitbottomskip=\topskip, %
}% 

\mdfdefinestyle{notation-frame}{
linewidth=10pt, %
linecolor= blue!30, % 
backgroundcolor= blue!10,
topline=false, %
bottomline=false, %
rightline=false,%
leftmargin=0pt, %
innerleftmargin=15pt, %
innerrightmargin=1em, 
rightmargin=0pt, % 
innertopmargin=-2pt,%
innerbottommargin=6pt, % 
splittopskip=\topskip, %
%splitbottomskip=\topskip, %
}% 

\mdfdefinestyle{remarque-frame}{
linewidth=0pt, %
linecolor= black!0, % 
backgroundcolor= blue!0,
topline=false, %
bottomline=false, %
rightline=false,%
leftmargin=0pt, %
innerleftmargin=25pt, %
innerrightmargin=25pt, 
rightmargin=0pt, % 
innertopmargin=-2pt,%
innerbottommargin=6pt, % 
splittopskip=\topskip, %
%splitbottomskip=\topskip, %
}% 
\surroundwithmdframed[style=defi-frame]{defi}

\surroundwithmdframed[style=thm-frame]{thm}

\surroundwithmdframed[style=lem-frame]{lem}

\surroundwithmdframed[style=notation-frame]{notation}

\surroundwithmdframed[style=remarque-frame]{remarque}

\surroundwithmdframed[style=remarque-frame]{remarques}

\surroundwithmdframed[style=methode-frame]{methode}

\theoremstyle{definition-style}

%pour avoir un compteur commun a remarque et remarques
\newcounter{compteurremarque}[chapter]
\renewcommand{\thecompteurremarque}{\arabic{compteurremarque}} %définition de l'affichage du compteur

\newtheorem{defi}{Définition}[chapter]
\renewcommand{\thedefi}{\arabic{defi}} %pour ne pas avoir le numéro du chapitre dans la numération de la def
\newtheorem{thm}{Théorème}[chapter]
\renewcommand{\thethm}{\arabic{thm}}
\newtheorem{lem}{Lemme}[chapter]
\renewcommand{\thelem}{\arabic{lem}}
\newtheorem{notation}{Notation}[chapter]
\renewcommand{\thenotation}{\arabic{notation}}
%\newtheorem{remarque}{Remarque}[chapter]
%\renewcommand{\theremarque}{\arabic{remarque}}
%\newtheorem{remarques}{Remarques}[chapter]
%\renewcommand{\theremarques}{\arabic{remarques}}
\newtheorem{methode}{Méthode}[chapter]
\renewcommand{\themethode}{\arabic{methode}}


\newtheorem{remarque}[compteurremarque]{Remarque}
\newtheorem{remarques}[compteurremarque]{Remarques} %remarque et remarques prennent compteurremarque comme compteur commun

\lstMakeShortInline[columns=fixed]|

%%%%%%%% Commandes de Matthieu

%-------------------------------------------------------------------------------
%---- Eclairage : en encadré sur fond jaune avec symbôle "ampoule" à gauche ----
%-------------------------------------------------------------------------------
\definecolor{coleclairage}{RGB}{255 , 221 , 156}
\definecolor{contoureclairage}{RGB}{255 , 192 , 0}
\newenvironment{eclairage}
{
	\begin{center}%
		\begin{tikzpicture}%
			\node[rectangle, draw=contoureclairage, top color=coleclairage!50, bottom color=coleclairage!140, rounded corners=5pt, inner xsep=5pt, inner ysep=6pt, outer ysep=10pt]\bgroup                     
			\begin{minipage}{0.98\linewidth}
				\begin{minipage}{0.08\linewidth}\centerline{\includegraphics[scale=1]{images/Symbole_eclairage.png}}\end{minipage}
				\begin{minipage}{0.89\linewidth}\itshape\footnotesize
				}
				{                		
				\end{minipage}
			\end{minipage}\egroup;%
		\end{tikzpicture}%
	\end{center}%
}



\newcommand{\putfigure}[6]{
    \begin{minipage}{#1\textwidth}
    \centering
        \includegraphics[width=#2\textwidth,height=#3\textheight]{#4}
        \captionof{figure}{#5}
        \label{#6}
    \end{minipage}
}


\newcommand{\itemb}[1]{\item \textbf{#1}}



%-------------------------------------------------------------------------------
%---- apprendre : en encadré sur fond jaune avec symbole "ampoule" à gauche ----
%-------------------------------------------------------------------------------
\definecolor{colapprendre}{RGB}{50,205,50}
\definecolor{contourapprendre}{RGB}{34,139,34}
\newenvironment{apprendre}
{
	\begin{center}%
		\begin{tikzpicture}%
			\node[rectangle, draw=contourapprendre, top color=colapprendre!10, bottom color=colapprendre!50, rounded corners=5pt, inner xsep=5pt, inner ysep=6pt, outer ysep=10pt]\bgroup                     
			\begin{minipage}{0.98\linewidth}
				\begin{minipage}{0.08\linewidth}\centerline{\includegraphics[width=30px]{images/Symbole_learn.png}}\end{minipage}
				\begin{minipage}{0.89\linewidth}\itshape\footnotesize
				}
				{                		
				\end{minipage}
			\end{minipage}\egroup;%
		\end{tikzpicture}%
	\end{center}%
}

\definecolor{colimportant}{RGB}{247 , 189 , 164}
\definecolor{contourimportant}{RGB}{237 , 125 , 49}
\newenvironment{important}
{
	\begin{center}%
		\begin{tikzpicture}%
			\node[rectangle, draw=contourimportant, top color=colimportant!50, bottom color=colimportant!140, rounded corners=5pt, inner xsep=5pt, inner ysep=6pt, outer ysep=10pt]\bgroup                     
			\begin{minipage}{0.08\linewidth}\centerline{\includegraphics[scale=0.8]{images/Symbole_attention.png}}\end{minipage}
			\begin{minipage}{0.89\linewidth}
			}
			{                		
			\end{minipage}\egroup;
		\end{tikzpicture}%
	\end{center}%
}

%----------------------------------------------
\newcounter{compteurmonprogramme}
\setcounter{compteurmonprogramme}{1}
\newenvironment{monprogramme}
{
	\begin{center}%
		\begin{tikzpicture}%
			\node[rectangle, draw=black, rounded corners=5pt, inner xsep=5pt, inner ysep=6pt, outer ysep=10pt]\bgroup                     
			\begin{minipage}{0.98\linewidth}
				{\bf Programme \thechapter.\thecompteurmonprogramme\;:}
				
				\vspace{2mm}\hspace{7.5mm}
				\begin{minipage}{0.93\linewidth}
				}
				{                		
				\end{minipage}
				\stepcounter{compteurmonprogramme}
			\end{minipage}\egroup;%
		\end{tikzpicture}%
	\end{center}%
}



%------------------------------------------------
%---- Définition : en encadré sur fond blanc ----
%------------------------------------------------
\newcounter{compteurdef}
\setcounter{compteurdef}{1}
\newenvironment{mydefinition}
{
	\begin{center}%
	\begin{tikzpicture}%
		\node[rectangle, draw=black, rounded corners=5pt, inner xsep=5pt, inner ysep=6pt, outer ysep=10pt]\bgroup                    
		\begin{minipage}{0.98\linewidth}
			{\bf {\underline {Définition \thechapter.\thecompteurdef}}}
			
			\vspace{2mm}\hspace{4.5mm}
			\begin{minipage}{0.93\linewidth}
			}
			{                		
			\end{minipage}
			\stepcounter{compteurdef}
		\end{minipage}\egroup;%
	\end{tikzpicture}%
\end{center}%
}

\newenvironment{mydefinitions}
{
	\begin{center}%
		\begin{tikzpicture}%
			\node[rectangle, draw=black, rounded corners=5pt, inner xsep=5pt, inner ysep=6pt, outer ysep=10pt]\bgroup                    
			\begin{minipage}{0.98\linewidth}
				{\bf {\underline {Définitions \thechapter.\thecompteurdef}}}
				
				\vspace{3mm}\hspace{4.5mm}
				\begin{minipage}{0.93\linewidth}\slshape
					\begin{itemize}\setlength{\itemsep}{1mm}
					}
					{                		
				\end{itemize}\end{minipage}
				\stepcounter{compteurdef}
			\end{minipage}\egroup;%
		\end{tikzpicture}%
	\end{center}%
}



%------------------------------------------------
%---- Exemple : en encadré sur fond blanc ----
%------------------------------------------------
\newcounter{compteurex}
\setcounter{compteurex}{1}
\newenvironment{myexample}{
	\begin{center}
	\vspace{-3mm}
	\begin{minipage}{1\linewidth}
		\vspace{2mm}
		{\textsl {\underline {Exemple \thechapter.\thecompteurex}}}
		
		\vspace{2mm}\hspace{2.5mm}
		\begin{minipage}{1\linewidth}
			\begin{mdframed}[topline=false,rightline=false,bottomline=false]
}
{
			\end{mdframed}
		\end{minipage}
		\stepcounter{compteurex}
	\end{minipage}
	\end{center}
}
\newenvironment{myexamples}{
	\begin{center}
		\vspace{-3mm}
		\begin{minipage}{1\linewidth}
			\vspace{2mm}
			{\textsl {\underline {Exemples \thechapter.\thecompteurex}}}
			
			\vspace{2mm}\hspace{2.5mm}
			\begin{minipage}{1\linewidth}%\slshape
				\begin{mdframed}[topline=false,rightline=false,bottomline=false]
					\begin{enumerate}\setlength{\itemsep}{1mm}
				}
				{
					\end{enumerate}
				\end{mdframed}
			\end{minipage}
			\stepcounter{compteurex}
		\end{minipage}
	\end{center}
}



\newtheorem*{question}{Question}



%%%%%%%%%%%%%%%%%%%%%%%%%%%%%%%%%%%%%%%%%%%%%%%%%%%%%%%%%%%%%%%%%%%%%%%%%%%%
%%%%%%%%%%%%%%%%%%%%%%%%%%%   LABELS activite   %%%%%%%%%%%%%%%%%%%%%%%%%%%
%%%%%%%%%%%%%%%%%%%%%%%%%%%%%%%%%%%%%%%%%%%%%%%%%%%%%%%%%%%%%%%%%%%%%%%%%%%%
\newcommand{\act}{\textbf{\textsl{Activité \arabic{compteuract}}} \vspace*{1mm}\\ \addtocounter{compteuract}{1}}
\newcommand{\actnomme}[1]{{\bf Activité \arabic{compteuract} {\textsl{\small (#1)}}}\vspace*{1mm}\\  \addtocounter{compteuract}{1}}
\newcounter{compteuract}
\setcounter{compteuract}{1}
\newcommand{\getactcompteur}{{\the\numexpr \arabic{compteuract} - 1 \relax}}



%%%%%%%%%%%%%%%%%%%%%%%%%%%%%%%%%%%%%%%%%%%%%%%%%%%%%%%%%%%%%%%%%%%%%%%%%%%%
%%%%%%%%%%%%%%%%%%%%%%%%%%%   LABELS EXERCICES   %%%%%%%%%%%%%%%%%%%%%%%%%%%
%%%%%%%%%%%%%%%%%%%%%%%%%%%%%%%%%%%%%%%%%%%%%%%%%%%%%%%%%%%%%%%%%%%%%%%%%%%%
\newcommand{\exo}{\textbf{\textsl{Exercice \arabic{compteurexo}}} \vspace*{1mm}\\ \addtocounter{compteurexo}{1}}
\newcommand{\exonomme}[1]{{\bf Exercice \arabic{compteurexo} {\textsl{\small (#1)}}}\vspace*{1mm}\\  \addtocounter{compteurexo}{1}}
\newcommand{\eexo}{\vspace{5mm}} % espace pour séparer les exercices
\newcounter{compteurexo}
\setcounter{compteurexo}{1}
\newcommand{\getexocompteur}{{\the\numexpr \arabic{compteurexo} - 1 \relax}}


\begin{document}
%		\title{\vspace{-3cm}Série 1}
%		\date{\vspace{-2cm}}
%		\maketitle

\fancyhead[CO,CE]{\sc Série \arabic{section} \hspace{0.5mm}}

\setcounter{section}{\numero}
\section{Types et opérations}				
\Opensolutionfile{mycor}[cor01]

%%%%%%%%%%%%%%%%%%%%%%%%%%%%%%%% E X E R C I C E  %%%%%%%%%%%%%%%%%%%%%%%%%%%%%%%%%%%
\exo{}  ~\\ 
Répondre aux questions suivantes et ensuite les vérifier dans la console Python.

\begin{enumerate}
	\item De quel type est le résultat du calcul 72/3 en Python ?
	\item Quelle est la valeur des expressions suivantes:  
	   \begin{lstlisting}[numbers=none]
>>> (2 or 0) and (0.1 == 1/10)
>>> (2 or 0) and (0.1 == (0.4-0.3))
>>> False or not (not True)
       \end{lstlisting}
       Pouvez vous expliquer la différence entre la première et la deuxième expression ?
	\end{enumerate}

	\begin{correction}
		~\\ 
		\begin{enumerate}
	\item \lstinline{float} même si 72 et 3 sont du type \lstinline{int}. Pour vérifier: \lstinline{>>> type(72/3)}
	\item \lstinline{>>> True}\newline
	      \lstinline{>>> False}\newline
	      \lstinline{>>> True}\newline
	      Le problème est que le nombre \lstinline{0,4} est une approximation et sa soustraction avec \lstinline{0,3} donne une valeur \lstinline{> 0,1}.
 		\end{enumerate}
	\end{correction}
\finexo
%%%%%%%%%%%%%%%%%%%%%%%%%%%%%%%% E X E R C I C E %%%%%%%%%%%%%%%%%%%%%%%%%%%%%%%%%%%
\exo{}  ~\\ 
Tester dans la console \py les commandes suivantes:
\begin{lstlisting}[numbers=none]
>>> type(123)
>>> type(123.)
>>> type("blabla")
>>> type(True)
\end{lstlisting}

Indiquer lesquels des objets suivants sont valides en Python et le cas échéant de quel type d'objet il s'agit :
\begin{multicols}{2}
	\begin{enumerate}[label=\alph*)]
		\item "rewr"
		\item gdru
		\item 5
		\item 5.0
		\item 5,0
		\item 'sism2021'
		\item 18gd
		\item "65.5"
		\item True
		\item "False"
	\end{enumerate}
\end{multicols}
	\begin{correction}
	~\\ 
	Utiliser la fonction \lstinline{type()} dans l'interpréteur \py pour vérifier vos réponses.
	\begin{multicols}{2}
	\begin{enumerate}[label=\alph*)]
		\item "rewr": str
		\item Erreur grdu n'est pas défini.
		\item 5: int
		\item 5.0: float
		\item 5,0: n'est pas un float mais un tuple (sera étudié plus tard)
		\item 'sism2021': est un str
		\item 18gd: invalide syntaxe
		\item "65.5": str
		\item True: : bool (booléen)
		\item "False": str
	\end{enumerate}
\end{multicols}
\end{correction}
\finexo

%%%%%%%%%%%%%%%%%%%%%%%%%%%%%%%% E X E R C I C E %%%%%%%%%%%%%%%%%%%%%%%%%%%%%%%%%%%
\exo{}[Convertir dans un type]  ~\\ 
 En s'aidant de la console \py, essayez de déterminer le rôle des fonctions \lstinline{int()},
\lstinline{float()} et \lstinline{str()} en écrivant par exemple les lignes suivantes :
\begin{lstlisting}[numbers=none]
>>> int(3.6)
>>> float(3)
>>> str(3)
>>> 2**1000
>>> float(2**1000)
>>> int('3')
\end{lstlisting}
\vspace{0.3cm}
\textit{Indication : Vous pouvez vous aider de la commande \lstinline{type()} pour analyser le résultat des fonctions \lstinline{int()}, \lstinline{float()} et \lstinline{str()}.}
	\begin{correction}
	~\\ 
	Les fonctions \lstinline{int()}, \lstinline{float()} et \lstinline{str()} permettent de convertir les données dans le type indiqué par la fonction.
\end{correction}
\finexo
\newpage
%%%%%%%%%%%%%%%%%%%%%%%%%%%%%%%% E X E R C I C E %%%%%%%%%%%%%%%%%%%%%%%%%%%%%%%%%%%
\exo{}[Expression et priorité des opérations]  ~\\ 
Prédire le résultat des instructions suivantes en \py:
\begin{multicols}{2}
	\begin{enumerate}[label=\alph*)]
		\item \lstinline{3+2*5}
		\item \lstinline{2*5**2}
		\item \lstinline{(2*5)**2}
		\item \lstinline{6/4*2}
		\item \lstinline{6.0/4*2}
		\item \lstinline{7 + 3 * 6 / 2 - 1}
		\item \lstinline{2 % 2 + 2 * 2 - 2 / 2}
		\item \lstinline{3 * 9 * (3 + (9 * 3 / (3)))}
	\end{enumerate}
\end{multicols}
	Entrer ces expressions dans la console \py pour vérifier les réponses.
	
    \begin{correction}
	~\\ 
\begin{multicols}{2}
	\begin{enumerate}[label=\alph*)]
		\item \lstinline{13}
		\item \lstinline{50}
		\item \lstinline{100}
		\item \lstinline{3.0}
		\item \lstinline{3.0}
		\item \lstinline{15.0}
		\item \lstinline{3.0}
		\item \lstinline{324.0}
		\end{enumerate}
	\end{multicols}
\end{correction}
\finexo

%%%%%%%%%%%%%%%%%%%%%%%%%%%%%%%% E X E R C I C E %%%%%%%%%%%%%%%%%%%%%%%%%%%%%%%%%%%
\exo{}  ~\\ 
Parmi les propositions, laquelle n'est pas une expression ?
\begin{multicols}{2}
	\begin{enumerate}[label=\alph*)]
		\item \lstinline{a < b}
		\item \lstinline{a != b}
		\item \lstinline{a = b}
		\item \lstinline{a >= b}
	\end{enumerate}
\end{multicols}
\begin{correction}
	~\\ 
	c) a = b n'est pas une expression, il s'agit d'une affectation. La valeur de b est affectée à la variable a.
\end{correction}
\finexo
%%%%%%%%%%%%%%%%%%%%%%%%%%%%%%%% E X E R C I C E %%%%%%%%%%%%%%%%%%%%%%%%%%%%%%%%%%%

\exo{}  ~\\ 
On considère les instructions suivantes exécutées dans l'ordre.
Quel est le résultat affiché de la commande: \lstinline{print(a==b+1)}?
\begin{multicols}{2}
	\begin{enumerate}[label=\alph*)]
		\item \lstinline{8}
		\item \lstinline{Aucun, une erreur est signalée.}
		\item \lstinline{False}
		\item \lstinline{True}
	\end{enumerate}
\end{multicols}
\begin{lstlisting}[numbers=none]
>>> a=8
>>> b=5
>>> a==b+1
>>> b=b+1
>>> a==b+1
>>> b=b+1
>>> print(a==b+1)
\end{lstlisting}
\begin{correction}
	~\\ 
	d) \lstinline{True}
\end{correction}
\finexo
%%%%%%%%%%%%%%%%%%%%%%%%%%%%%%%% E X E R C I C E %%%%%%%%%%%%%%%%%%%%%%%%%%%%%%%%%%%
\exo{}  ~\\ 
On considère les deux instructions a=a+b et b=a-b exécutées dans l'ordre. Quelle affirmation est exacte ?
	\begin{enumerate}[label=\alph*)]
		\item Si les valeurs initiales de a et b son respectivement "bon" et "jour", alors le programme est interrompu par une erreur.
		\item Si les valeurs initiales de a et b son respectivement "bon" et "jour", alors les valeurs finales (des variables a et b) sont respectivement "bonjour" et "bon".
		\item Si les valeurs initiales de a et b son respectivement 5 et 2, alors les valeurs finales (des variables a et b) sont respectivement 7 et 3.
		\item Si les valeurs initiales de a et b son respectivement 5 et 2, alors les valeurs finales (des variables a et b) sont respectivement 7 et 2.
	\end{enumerate}

\begin{correction}
	~\\ 
	L'affirmation a) est correct car la soustraction (-) ne fait pas partie des opérations permises avec des chaînes de caractères.
\end{correction}
\finexo
%%%%%%%%%%%%%%%%%%%%%%%%%%%%%%%% E X E R C I C E %%%%%%%%%%%%%%%%%%%%%%%%%%%%%%%%%%%
\exo{}  ~\\ 
Manipulations fondamentales sur les chaînes. On définit les deux suivantes:
\begin{lstlisting}[numbers=none]
>>> ch1 = "Ma première chaîne de caractères"
>>> ch2 = "en Python"
\end{lstlisting}
Répondez, uniquement à l'aide de \lstinline{ch1} et \lstinline{ch2}, aux questions suivantes. Donnez le code \py correspondant:
	\begin{enumerate}[label*=\arabic*] 
		\item Quel est le nombre de caractères de \lstinline{ch1} ?
		\item Créer la chaîne \lstinline{ch3} égale à "première".
		\item Créer la chaîne \lstinline{ch4} égale à "Ma première chaîne de caractères en Python".
		\item Créer la chaîne \lstinline{ch5} égale à "Mon premier".
	\end{enumerate}

\begin{correction}
	~\\ 
	\begin{enumerate}[label*=\arabic*] 
		\item \lstinline{>>> len(ch1)         # renvoie 32}
		\item \lstinline{>>> ch3 = ch1[3:11]}
		\item \lstinline{>>> ch4 = ch1 + ch1[2]+ ch2    # Ne pas oublier l'espace}
		\item 
		     \lstinline{>>> l2 = len(ch2)   # pour éviter de compter les indices dans ch2}\newline
		     \lstinline{>>> ch5 = ch1[0]+ch2[l2-2:]+ch1[2:8]+ch1[5]+ch1[4]}
	\end{enumerate}
\end{correction}
\finexo

% Solution		
		\newpage
		\setcounter{page}{1}
		\setcounter{section}{\numero}
		\Closesolutionfile{mycor}
		\titleformat{\section}[hang]{\Large \bfseries}{Corrigé Série \thesection:\ }{0pt}{}
		

		\fancyhead[CO,CE]{\sc Corrigé Série \arabic{section} \hspace{0.5mm}}
		\section{}
		\Readsolutionfile{mycor}
	\end{document}