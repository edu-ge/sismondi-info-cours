\usepackage[french]{babel}
\usepackage[utf8]{inputenc}
\usepackage{answers}

\usepackage{hyperref}
\usepackage{multicol}

\usepackage[table,xcdraw]{xcolor}
\usepackage{listings}
\definecolor{ForestGreen}{RGB}{34,139,34}


\usepackage{enumitem}

\AtBeginDocument{\def\labelitemi{$\bullet$}}


\newcommand{\py}{\lstinline{Python} }


\definecolor{backcolour}{rgb}{0.95,0.95,0.92}

\lstset{%
	language         = Python,
	backgroundcolor  = \color{backcolour},
	basicstyle       = \ttfamily, % \upshape\ttfamily,
	keywordstyle     = \bfseries\color{blue}, %\bfseries,
	stringstyle      = \color{magenta},
	commentstyle     = \color{ForestGreen},
	alsoletter = > ,
	morekeywords = {>>>,as,assert,False,None, nonlocal,True, with,yield , <<, >>, :},
	showstringspaces = false,
	numbers=left,
	stepnumber=1,
	literate={à}{{\`{a}}}1 {é}{{\'e}}1 {è}{{\`{e}}}1 {ê}{{\^{e}}}1 {Ê}{{\^{E}}}1 {î}{{\^i}}1 {ô}{{\^{o}}}1 {ç}{{\c{c}}}1 {Ç}{{\c{C}}}1
}

\newcommand{\itemb}[1]{\item \textbf{#1}}

\usepackage{fancyhdr}  %package pour en-tetes et pied de pages
\usepackage{sectsty} % Permet de faire des modifications de police dans diverses sections des "headings" (cf. modif presentation de la page)
\pagestyle{fancy}       %Style pour en-tetes et pieds de pages
\fancyhead[CO,CE]{\sc Série 1\hspace{0.5mm}}
\fancyhead[RO,LE]{Collège Sismondi}  % LaTeX/TEX define \strut to be an invisible box of width zero that extends just enough above and below the baseline. Cela permet d'augementer légèrement la taille en bas de la box de manière à ce qu'elle soit collée à la ligne.
\fancyhead[LO,RE]{\small\ \textsl{1\textsuperscript{ère} année - DO Informatique}}
\fancyfoot[RO,LE]{2021 - 2022}
\fancyfoot[LO,RE]{\small }
\fancyfoot[CO,CE]{\thepage}

\fancyhfoffset[l]{1.2cm} % le "l" en paramètre permet d'indiquer qu'on ne veut modifier que la marge à gauche.
\renewcommand{\headrule}{{%
		\hrule \headwidth \headrulewidth \vskip-\headrulewidth}}
\renewcommand\footrulewidth{\headrulewidth}
\renewcommand{\footrule}{{%
		\vskip-\footruleskip\vskip-\footrulewidth
		\hrule \headwidth \footrulewidth\vskip\footruleskip}}

\usepackage{tikz}
%-------------------------------------------------------------------------------
%---- Eclairage : en encadré sur fond jaune avec symbôle "ampoule" à gauche ----
%-------------------------------------------------------------------------------
\definecolor{coleclairage}{RGB}{255 , 221 , 156}
\definecolor{contoureclairage}{RGB}{255 , 192 , 0}
\newenvironment{eclairage}
{
	\begin{center}%
		\begin{tikzpicture}%
			\node[rectangle, draw=contoureclairage, top color=coleclairage!50, bottom color=coleclairage!140, rounded corners=5pt, inner xsep=5pt, inner ysep=6pt, outer ysep=10pt]\bgroup                     
			\begin{minipage}{0.98\linewidth}
				\begin{minipage}{0.08\linewidth}\centerline{\includegraphics[scale=1]{Symbole_eclairage.png}}\end{minipage}
				\begin{minipage}{0.89\linewidth}\itshape\footnotesize
				}
				{                		
				\end{minipage}
			\end{minipage}\egroup;%
		\end{tikzpicture}%
	\end{center}%
}

%-------------------------------------------------------------------------------
%---- apprendre : en encadré sur fond jaune avec symbôle "ampoule" à gauche ----
%-------------------------------------------------------------------------------
\definecolor{colapprendre}{RGB}{50,205,50}
\definecolor{contourapprendre}{RGB}{34,139,34}
\newenvironment{apprendre}
{
	\begin{center}%
		\begin{tikzpicture}%
			\node[rectangle, draw=contourapprendre, top color=colapprendre!10, bottom color=colapprendre!50, rounded corners=5pt, inner xsep=5pt, inner ysep=6pt, outer ysep=10pt]\bgroup                     
			\begin{minipage}{0.98\linewidth}
				\begin{minipage}{0.08\linewidth}\centerline{\includegraphics[width=30px]{Symbole_learn.png}}\end{minipage}
				\begin{minipage}{0.89\linewidth}\itshape\footnotesize
				}
				{                		
				\end{minipage}
			\end{minipage}\egroup;%
		\end{tikzpicture}%
	\end{center}%
}

\definecolor{colimportant}{RGB}{247 , 189 , 164}
\definecolor{contourimportant}{RGB}{237 , 125 , 49}
\newenvironment{important}
{
	\begin{center}%
		\begin{tikzpicture}%
			\node[rectangle, draw=contourimportant, top color=colimportant!50, bottom color=colimportant!140, rounded corners=5pt, inner xsep=5pt, inner ysep=6pt, outer ysep=10pt]\bgroup                     
			\begin{minipage}{0.08\linewidth}\centerline{\includegraphics[scale=0.8]{Symbole_attention.png}}\end{minipage}
			\begin{minipage}{0.89\linewidth}
			}
			{                		
			\end{minipage}\egroup;
		\end{tikzpicture}%
	\end{center}%
}

%-----------------------------------------------------------------
%---- Modification présentation de la page: marges de la page ----
%-----------------------------------------------------------------
%\addtolength{\hoffset}{-1in}              % 1
%\addtolength{\voffset}{-1in}              % 2
\addtolength{\oddsidemargin}{-0.1 in} % 3
\addtolength{\evensidemargin}{-1in} % 3
\addtolength{\topmargin}{-1in}       % 4
\addtolength{\headheight}{6pt}       % 5
%\addtolength{\headsep}{-0.2cm}           % 6
\setlength{\textheight}{26cm}    % 7
\setlength{\textwidth}{16.5cm}      % 8
\addtolength{\marginparsep}{0pt}      % 9
\setlength{\marginparwidth}{0pt}   % 10
\addtolength{\footskip}{-1mm}           %11

\setlength{\parindent}{0em}% pas d'indentation


% Customiser le nom des sections
\usepackage{titlesec}
\titleformat{\section}[hang]{\Large \bfseries}{Série \thesection:\ }{0pt}{}

\renewcommand{\familydefault}{\sfdefault} % pour avoir des polices san serif

\newtheorem{Exc}{Exercice}
\Newassociation{correction}{Soln}{mycor}
\renewcommand{\Solnlabel}[1]{\bfseries Ex #1 }
\def\exo#1{%
	\futurelet\testchar\MaybeOptArgmyexoo}
\def\MaybeOptArgmyexoo{
	\ifx[\testchar \let\next\OptArgmyexoo
	\else \let\next\NoOptArgmyexoo \fi \next}
\def\OptArgmyexoo[#1]{%
	\begin{Exc}[#1]\normalfont}
	\def\NoOptArgmyexoo{%
		\begin{Exc}\normalfont}
		\newcommand{\finexo}{\end{Exc} \vspace{3mm}}
	\newcommand{\flag}[1]{}
	\newcommand{\entete}[1]

\newcommand{\getexocompteur}{{\the\numexpr \arabic{Exc}  \relax}}	
	
\newcommand{\eexo}{\vspace{5mm}} % espace pour séparer les exercices