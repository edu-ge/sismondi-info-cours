\documentclass[a4paper,12pt]{article} 

%%%%%%%%%%%%%%%%%%%%%%%%%%%%%%%% CONSTANTES %%%%%%%%%%%%%%%%%%%%%%%%%%%%%%%%%%%
\newcommand{\numero}{7}                                    %Numéro de la série -1

% Ressources à installer : texlive-science et texlive-lang-french ou texlive-lang-european
%à compiler avec XeLaTeX à cause des image pstricks

\usepackage[utf8]{inputenc}
\usepackage[T1]{fontenc}

\usepackage[french]{babel}
\usepackage{listingsutf8}
\usepackage{tikz,pgf}
\usepackage{amsthm,titlesec}
\usepackage{mdframed} % pour le format des définition, théorèmes...
\usepackage{xcolor,rotating,systeme}
\usepackage[Glenn]{fncychap} %pour le format des chapitres
\usepackage[top=3cm,bottom=3cm,left=3cm,right=2cm,headsep=10pt,a4paper]{geometry} % Page margins
\usepackage{multicol}
\usepackage{enumerate,cancel}
\usepackage{lipsum,minitoc}
%\usepackage{chngcntr}
\usepackage{amsmath,amsfonts,minitoc,amsthm,pgfplots}
\usepackage{mathrsfs}
\usetikzlibrary{arrows, intersections}
\usepackage{mathrsfs}
\usepackage{array,yhmath}
\usepackage{hyperref}
\usepackage{caption}
\usepackage{float}
\usepackage{wrapfig}
\restylefloat{figure}
\usepackage{subfig}

\usepackage{listings}
\usepackage{color}
\lstset{ % General setup for the package
    language=Python,
    basicstyle=\footnotesize,
    basewidth=0.7em,
    numbers=left,
     numberstyle=\tiny,
    frame=lines,
    tabsize=4,
    columns=fixed,
    showstringspaces=false,
    showtabs=false,
    keepspaces,
    commentstyle=\color{gray},
    keywordstyle=\color{blue},
    stringstyle=\color{magenta},
    inputencoding=utf8/latin1,
            extendedchars=true,
            literate=%
            {é}{{\'{e}}}1
            {è}{{\`{e}}}1
            {ê}{{\^{e}}}1
            {ë}{{\¨{e}}}1
            {û}{{\^{u}}}1
            {ù}{{\`{u}}}1
            {â}{{\^{a}}}1
            {à}{{\`{a}}}1
            {î}{{\^{i}}}1
            {ô}{{\^{o}}}1
            {ç}{{\c{c}}}1
            {Ç}{{\c{C}}}1
            {É}{{\'{E}}}1
            {Ê}{{\^{E}}}1
            {À}{{\`{A}}}1
            {Â}{{\^{A}}}1
            {Î}{{\^{I}}}1
    }



\usepackage{lipsum}
\newtheorem{example}{Exemple}

\usepackage{pstricks,pst-plot,pstricks-add}
\usepackage{pst-func,tkz-tab,tkz-euclide}
\usepackage{graphicx}
%macro exo
\newcounter{mtex}[chapter]
\newcommand{\mtexlabel}{\cadrexo{\themtex}}
\newcommand{\cadrexo}[1]{%
\tikz\node[rectangle,minimum size=6mm,rounded corners=2mm,fill=ocre,inner sep=0pt,text width=0.8cm,align=center]{\large\bfseries\textcolor{white}{#1}};}
\definecolor{ocre}{RGB}{196,106,106}
\definecolor{vert}{RGB}{55,120,0}
\newcommand{\mtexlabelpos}[1]{%
\makebox[0pt][r]{\raisebox{#1\baselineskip}[0pt][0pt]{\mtexlabel\quad}}}
\newenvironment{exercice}[1][\empty]{%
\refstepcounter{mtex}%
\trivlist\item\relax%  
\ifx#1\empty\mtexlabelpos{-.7}\else\mtexlabelpos{-.5}%
\hfill\textbf{#1}\hfill\mbox{}\par\fi%
}{\endtrivlist}

\newcommand{\N}{\mathbb{N}}
\newcommand{\Z}{\mathbb Z}
\newcommand{\Q}{\mathbb Q}
\newcommand{\R}{\mathbb R}
\newcommand{\C}{\mathbb C}

\newcommand{\ssi}{\;\;\Leftrightarrow\;\;}

\newcommand{\point}[1]{-- +(#1,#1) -- +(-#1,-#1) -- +(0,0) -- +(-#1,#1) -- +(#1,-#1)} 


\newcommand{\parallele}{\mathbin{\!/\mkern-5mu/\!}} %signe parallèle

\titleformat{\chapter}[frame]
{\setlength\fboxrule{2.25pt}\color{black}}%
{\filleft\scshape\LARGE%
\enspace  Chapitre \thechapter\enspace}%
{20pt}
{\rule{0pt}{30pt}\Huge\scshape\filleft}

\newcommand\fonction[5]{
$\begin{array}{rrll}
#1: & #2 & \rightarrow & #3 \\
  & #4 & \mapsto & #5
\end{array}$
}

%\titleformat
%{\section} % command
%[hang] % shape
%{\bfseries\Large} % format
%{ \thesection\quad} % label
%{0em} % sep
%{
%} % before-code
%[
%\vspace{-1ex}%
%\textcolor{blue!60}{\rule{10cm}{3pt}}
%] % after-code


%\theoremstyle{break}
\newtheoremstyle{definition-style}  %style des theoreme
{}               
{}               
{}                   
{}                   
{\bf\sffamily} 
{~:\\[0.3cm]}                 
{.5em}               
{\thmname{#1}\thmnumber{ #2}\thmnote{ (#3)}}

\mdfdefinestyle{defi-frame}{
linewidth=10pt, %
linecolor= blue!50, % 
backgroundcolor= blue!20,
topline=false, %
bottomline=false, %
rightline=false,%
leftmargin=0pt, %
innerleftmargin=15pt, %
innerrightmargin=1em, 
rightmargin=0pt, % 
innertopmargin=-2pt,%
innerbottommargin=6pt, % 
splittopskip=\topskip, %
%splitbottomskip=\topskip, %
}% 

\mdfdefinestyle{methode-frame}{
	linewidth=10pt, %
linecolor= gray!70, % 
backgroundcolor= white,
topline=false, %
bottomline=false,%
rightline=false,%
leftmargin=0pt, %
innerleftmargin=15pt, %
innerrightmargin=1em, 
rightmargin=0pt, % 
innertopmargin=-2pt,%
innerbottommargin=6pt, % 
splittopskip=\topskip, %
%splitbottomskip=\topskip, %
}% 


\mdfdefinestyle{thm-frame}{
linewidth=10pt, %
linecolor= red!50, % 
backgroundcolor= orange!30,
topline=false, %
bottomline=false, %
rightline=false,%
leftmargin=0pt, %
innerleftmargin=15pt, %
innerrightmargin=1em, 
rightmargin=0pt, % 
innertopmargin=-2pt,%
innerbottommargin=6pt, % 
splittopskip=\topskip, %
%splitbottomskip=\topskip, %
}% 

\mdfdefinestyle{lem-frame}{
linewidth=10pt, %
linecolor= yellow!50, % 
backgroundcolor= yellow!10,
topline=false, %
bottomline=false, %
rightline=false,%
leftmargin=0pt, %
innerleftmargin=15pt, %
innerrightmargin=1em, 
rightmargin=0pt, % 
innertopmargin=-2pt,%
innerbottommargin=6pt, % 
splittopskip=\topskip, %
%splitbottomskip=\topskip, %
}% 

\mdfdefinestyle{notation-frame}{
linewidth=10pt, %
linecolor= blue!30, % 
backgroundcolor= blue!10,
topline=false, %
bottomline=false, %
rightline=false,%
leftmargin=0pt, %
innerleftmargin=15pt, %
innerrightmargin=1em, 
rightmargin=0pt, % 
innertopmargin=-2pt,%
innerbottommargin=6pt, % 
splittopskip=\topskip, %
%splitbottomskip=\topskip, %
}% 

\mdfdefinestyle{remarque-frame}{
linewidth=0pt, %
linecolor= black!0, % 
backgroundcolor= blue!0,
topline=false, %
bottomline=false, %
rightline=false,%
leftmargin=0pt, %
innerleftmargin=25pt, %
innerrightmargin=25pt, 
rightmargin=0pt, % 
innertopmargin=-2pt,%
innerbottommargin=6pt, % 
splittopskip=\topskip, %
%splitbottomskip=\topskip, %
}% 
\surroundwithmdframed[style=defi-frame]{defi}

\surroundwithmdframed[style=thm-frame]{thm}

\surroundwithmdframed[style=lem-frame]{lem}

\surroundwithmdframed[style=notation-frame]{notation}

\surroundwithmdframed[style=remarque-frame]{remarque}

\surroundwithmdframed[style=remarque-frame]{remarques}

\surroundwithmdframed[style=methode-frame]{methode}

\theoremstyle{definition-style}

%pour avoir un compteur commun a remarque et remarques
\newcounter{compteurremarque}[chapter]
\renewcommand{\thecompteurremarque}{\arabic{compteurremarque}} %définition de l'affichage du compteur

\newtheorem{defi}{Définition}[chapter]
\renewcommand{\thedefi}{\arabic{defi}} %pour ne pas avoir le numéro du chapitre dans la numération de la def
\newtheorem{thm}{Théorème}[chapter]
\renewcommand{\thethm}{\arabic{thm}}
\newtheorem{lem}{Lemme}[chapter]
\renewcommand{\thelem}{\arabic{lem}}
\newtheorem{notation}{Notation}[chapter]
\renewcommand{\thenotation}{\arabic{notation}}
%\newtheorem{remarque}{Remarque}[chapter]
%\renewcommand{\theremarque}{\arabic{remarque}}
%\newtheorem{remarques}{Remarques}[chapter]
%\renewcommand{\theremarques}{\arabic{remarques}}
\newtheorem{methode}{Méthode}[chapter]
\renewcommand{\themethode}{\arabic{methode}}


\newtheorem{remarque}[compteurremarque]{Remarque}
\newtheorem{remarques}[compteurremarque]{Remarques} %remarque et remarques prennent compteurremarque comme compteur commun

\lstMakeShortInline[columns=fixed]|

%%%%%%%% Commandes de Matthieu

%-------------------------------------------------------------------------------
%---- Eclairage : en encadré sur fond jaune avec symbôle "ampoule" à gauche ----
%-------------------------------------------------------------------------------
\definecolor{coleclairage}{RGB}{255 , 221 , 156}
\definecolor{contoureclairage}{RGB}{255 , 192 , 0}
\newenvironment{eclairage}
{
	\begin{center}%
		\begin{tikzpicture}%
			\node[rectangle, draw=contoureclairage, top color=coleclairage!50, bottom color=coleclairage!140, rounded corners=5pt, inner xsep=5pt, inner ysep=6pt, outer ysep=10pt]\bgroup                     
			\begin{minipage}{0.98\linewidth}
				\begin{minipage}{0.08\linewidth}\centerline{\includegraphics[scale=1]{images/Symbole_eclairage.png}}\end{minipage}
				\begin{minipage}{0.89\linewidth}\itshape\footnotesize
				}
				{                		
				\end{minipage}
			\end{minipage}\egroup;%
		\end{tikzpicture}%
	\end{center}%
}



\newcommand{\putfigure}[6]{
    \begin{minipage}{#1\textwidth}
    \centering
        \includegraphics[width=#2\textwidth,height=#3\textheight]{#4}
        \captionof{figure}{#5}
        \label{#6}
    \end{minipage}
}


\newcommand{\itemb}[1]{\item \textbf{#1}}



%-------------------------------------------------------------------------------
%---- apprendre : en encadré sur fond jaune avec symbole "ampoule" à gauche ----
%-------------------------------------------------------------------------------
\definecolor{colapprendre}{RGB}{50,205,50}
\definecolor{contourapprendre}{RGB}{34,139,34}
\newenvironment{apprendre}
{
	\begin{center}%
		\begin{tikzpicture}%
			\node[rectangle, draw=contourapprendre, top color=colapprendre!10, bottom color=colapprendre!50, rounded corners=5pt, inner xsep=5pt, inner ysep=6pt, outer ysep=10pt]\bgroup                     
			\begin{minipage}{0.98\linewidth}
				\begin{minipage}{0.08\linewidth}\centerline{\includegraphics[width=30px]{images/Symbole_learn.png}}\end{minipage}
				\begin{minipage}{0.89\linewidth}\itshape\footnotesize
				}
				{                		
				\end{minipage}
			\end{minipage}\egroup;%
		\end{tikzpicture}%
	\end{center}%
}

\definecolor{colimportant}{RGB}{247 , 189 , 164}
\definecolor{contourimportant}{RGB}{237 , 125 , 49}
\newenvironment{important}
{
	\begin{center}%
		\begin{tikzpicture}%
			\node[rectangle, draw=contourimportant, top color=colimportant!50, bottom color=colimportant!140, rounded corners=5pt, inner xsep=5pt, inner ysep=6pt, outer ysep=10pt]\bgroup                     
			\begin{minipage}{0.08\linewidth}\centerline{\includegraphics[scale=0.8]{images/Symbole_attention.png}}\end{minipage}
			\begin{minipage}{0.89\linewidth}
			}
			{                		
			\end{minipage}\egroup;
		\end{tikzpicture}%
	\end{center}%
}

%----------------------------------------------
\newcounter{compteurmonprogramme}
\setcounter{compteurmonprogramme}{1}
\newenvironment{monprogramme}
{
	\begin{center}%
		\begin{tikzpicture}%
			\node[rectangle, draw=black, rounded corners=5pt, inner xsep=5pt, inner ysep=6pt, outer ysep=10pt]\bgroup                     
			\begin{minipage}{0.98\linewidth}
				{\bf Programme \thechapter.\thecompteurmonprogramme\;:}
				
				\vspace{2mm}\hspace{7.5mm}
				\begin{minipage}{0.93\linewidth}
				}
				{                		
				\end{minipage}
				\stepcounter{compteurmonprogramme}
			\end{minipage}\egroup;%
		\end{tikzpicture}%
	\end{center}%
}



%------------------------------------------------
%---- Définition : en encadré sur fond blanc ----
%------------------------------------------------
\newcounter{compteurdef}
\setcounter{compteurdef}{1}
\newenvironment{mydefinition}
{
	\begin{center}%
	\begin{tikzpicture}%
		\node[rectangle, draw=black, rounded corners=5pt, inner xsep=5pt, inner ysep=6pt, outer ysep=10pt]\bgroup                    
		\begin{minipage}{0.98\linewidth}
			{\bf {\underline {Définition \thechapter.\thecompteurdef}}}
			
			\vspace{2mm}\hspace{4.5mm}
			\begin{minipage}{0.93\linewidth}
			}
			{                		
			\end{minipage}
			\stepcounter{compteurdef}
		\end{minipage}\egroup;%
	\end{tikzpicture}%
\end{center}%
}

\newenvironment{mydefinitions}
{
	\begin{center}%
		\begin{tikzpicture}%
			\node[rectangle, draw=black, rounded corners=5pt, inner xsep=5pt, inner ysep=6pt, outer ysep=10pt]\bgroup                    
			\begin{minipage}{0.98\linewidth}
				{\bf {\underline {Définitions \thechapter.\thecompteurdef}}}
				
				\vspace{3mm}\hspace{4.5mm}
				\begin{minipage}{0.93\linewidth}\slshape
					\begin{itemize}\setlength{\itemsep}{1mm}
					}
					{                		
				\end{itemize}\end{minipage}
				\stepcounter{compteurdef}
			\end{minipage}\egroup;%
		\end{tikzpicture}%
	\end{center}%
}



%------------------------------------------------
%---- Exemple : en encadré sur fond blanc ----
%------------------------------------------------
\newcounter{compteurex}
\setcounter{compteurex}{1}
\newenvironment{myexample}{
	\begin{center}
	\vspace{-3mm}
	\begin{minipage}{1\linewidth}
		\vspace{2mm}
		{\textsl {\underline {Exemple \thechapter.\thecompteurex}}}
		
		\vspace{2mm}\hspace{2.5mm}
		\begin{minipage}{1\linewidth}
			\begin{mdframed}[topline=false,rightline=false,bottomline=false]
}
{
			\end{mdframed}
		\end{minipage}
		\stepcounter{compteurex}
	\end{minipage}
	\end{center}
}
\newenvironment{myexamples}{
	\begin{center}
		\vspace{-3mm}
		\begin{minipage}{1\linewidth}
			\vspace{2mm}
			{\textsl {\underline {Exemples \thechapter.\thecompteurex}}}
			
			\vspace{2mm}\hspace{2.5mm}
			\begin{minipage}{1\linewidth}%\slshape
				\begin{mdframed}[topline=false,rightline=false,bottomline=false]
					\begin{enumerate}\setlength{\itemsep}{1mm}
				}
				{
					\end{enumerate}
				\end{mdframed}
			\end{minipage}
			\stepcounter{compteurex}
		\end{minipage}
	\end{center}
}



\newtheorem*{question}{Question}



%%%%%%%%%%%%%%%%%%%%%%%%%%%%%%%%%%%%%%%%%%%%%%%%%%%%%%%%%%%%%%%%%%%%%%%%%%%%
%%%%%%%%%%%%%%%%%%%%%%%%%%%   LABELS activite   %%%%%%%%%%%%%%%%%%%%%%%%%%%
%%%%%%%%%%%%%%%%%%%%%%%%%%%%%%%%%%%%%%%%%%%%%%%%%%%%%%%%%%%%%%%%%%%%%%%%%%%%
\newcommand{\act}{\textbf{\textsl{Activité \arabic{compteuract}}} \vspace*{1mm}\\ \addtocounter{compteuract}{1}}
\newcommand{\actnomme}[1]{{\bf Activité \arabic{compteuract} {\textsl{\small (#1)}}}\vspace*{1mm}\\  \addtocounter{compteuract}{1}}
\newcounter{compteuract}
\setcounter{compteuract}{1}
\newcommand{\getactcompteur}{{\the\numexpr \arabic{compteuract} - 1 \relax}}



%%%%%%%%%%%%%%%%%%%%%%%%%%%%%%%%%%%%%%%%%%%%%%%%%%%%%%%%%%%%%%%%%%%%%%%%%%%%
%%%%%%%%%%%%%%%%%%%%%%%%%%%   LABELS EXERCICES   %%%%%%%%%%%%%%%%%%%%%%%%%%%
%%%%%%%%%%%%%%%%%%%%%%%%%%%%%%%%%%%%%%%%%%%%%%%%%%%%%%%%%%%%%%%%%%%%%%%%%%%%
\newcommand{\exo}{\textbf{\textsl{Exercice \arabic{compteurexo}}} \vspace*{1mm}\\ \addtocounter{compteurexo}{1}}
\newcommand{\exonomme}[1]{{\bf Exercice \arabic{compteurexo} {\textsl{\small (#1)}}}\vspace*{1mm}\\  \addtocounter{compteurexo}{1}}
\newcommand{\eexo}{\vspace{5mm}} % espace pour séparer les exercices
\newcounter{compteurexo}
\setcounter{compteurexo}{1}
\newcommand{\getexocompteur}{{\the\numexpr \arabic{compteurexo} - 1 \relax}}



\begin{document}
%		\title{\vspace{-3cm}Série 1}
%		\date{\vspace{-2cm}}
%		\maketitle

\fancyhead[CO,CE]{\sc Série \arabic{section} \hspace{0.5mm}}

\setcounter{section}{\numero}
\section{Les variables}				
\Opensolutionfile{mycor}[cor_01]

\exo{}  ~\\ 
 Entourer en rouge les noms de variable entraînant une erreur d’exécution et en bleu les noms de variables n’entraînant pas d’erreurs d’exécution mais mal choisis d'après vous.
\begin{multicols}{7}
	{\small
		valeurs\\
		A'\\
		\_zarbi\\
		int\\
		beau\_pere\\
		nb\\
		beauperes\\
		Temp\_Corps \\
		x7\\
		temp-peau\\
		TC\\
		nb\_belle-meres\\
		belle fille\\
		nb\_beau\_fils\\
		Tableau\\
		False\_True\\
		1MA1\_DF07\\
		var\\
		n1
	}
\end{multicols}		
\finexo

\exo{}  ~\\ 
Proposez un nom de variable permettant de stocker :
\begin{itemize}
	\item le nombre de filles de en première OS Math Physique
	\item  le tarif d’un repas à la cafétéria
	\item l’aire d’un triangle (il n’y a qu’une seule figure)
	\item la note à une évaluation d’informatique
\end{itemize}
	\begin{correction}
		~\\ 
		Par exemple, dans les quatre cas:
		\begin{itemize}
			\item \lstinline{nb_filles_1er}
			\item  \lstinline{menu_prix}
			\item  \lstinline{aire_triangle}
			\item  \lstinline{note_eval_info}
		\end{itemize}
	\end{correction}
\finexo

\exo{}  ~\\ 
 Un prof de français à stocké les notes de deux de ses élèves dans les variable \lstinline{n1} et \lstinline{n2}. Le problème c'est qu'il s'est trompé, il doit maintenant permuter le contenu de ces deux variables.   Pour simplifier, on suppose que la variable  \lstinline{n1} stocke le nombre  \lstinline{3}, tandis que la variable  \lstinline{n2} stocke le nombre  \lstinline{5.5}.
\begin{center}
	\vspace{-5mm}
	\includegraphics[width=0.3\linewidth]{images/exo_permut.png}
\end{center}
\vspace{-5mm}
\begin{enumerate}
	\item Monsieur Alain Prolo propose l'algorithme suivant :
	\hspace{-8cm}
	\begin{minipage}[t]{0.85\linewidth}
		\begin{lstlisting}
n1 = 3
n2 = 5.5
n1 = n2
n2 = n1
		\end{lstlisting}
	\end{minipage}
	\begin{enumerate}[label=\alph*.]
		\item Compléter sur le tableau d'exécution ci-dessous, en donnant la valeur des variable \lstinline{n1} et \lstinline{n2} après l'exécution de chaque l'instruction numérotée.
		\begin{center}
			\begin{tabular}{|l|p{5cm}|p{5cm}|}
				\hline
				\rowcolor[HTML]{EFEFEF} 
				lignes & Valeur de n1  & Valeur de n2  \\ \hline
				1&         3 &         rien     \\ \hline
				2&          &              \\ \hline
				3&          &              \\ \hline
				4&          &              \\ \hline
				
			\end{tabular}
		\end{center}
		\item Est-ce le programme proposé par Monsieur Alain Prolot permet d'échanger les valeurs stockées dans les variables \lstinline{n1} et de \lstinline{n2} ?
	\end{enumerate}
	\newpage
	
	\item Madame Céline Rasteau propose l'algorithme suivant :
	\hspace{-8.8cm}
	\begin{minipage}[t]{0.85\linewidth}
		\begin{lstlisting}
n1 = 3
n2 = 5.5
n2 = n1
n1 = n2
		\end{lstlisting}
	\end{minipage}
	\begin{enumerate}[label=\alph*.]
		\item Compléter sur le tableau d'exécution ci-dessous, en donnant la valeur des variable \lstinline{n1} et \lstinline{n2} après l'exécution de chaque l'instruction numérotée.
		\begin{center}
			\begin{tabular}{|l|p{5cm}|p{5cm}|}
				\hline
				\rowcolor[HTML]{EFEFEF} 
				lignes & Valeur de n1  & Valeur de n2  \\ \hline
				1&          &              \\ \hline
				2&          &              \\ \hline
				3&          &              \\ \hline
				4&          &              \\ \hline
				
			\end{tabular}
		\end{center}
		\item Est-ce le programme proposé par madame Céline Rasteau permet d'échanger les valeurs stockées dans les variables \lstinline{n1} et de \lstinline{n2} ?		
	\end{enumerate}
	\item  Proposer un programme qui permet d'échanger les valeurs stockées dans les variables \textbf{sans utiliser l'affectation parallèle.}
\end{enumerate}

\begin{correction}
	~\\ 
	\begin{enumerate}
		\item 
		\begin{enumerate}[label=\alph*.]
			\item 
			\begin{center}
				\begin{tabular}{|l|p{5cm}|p{5cm}|}
					\hline
					\rowcolor[HTML]{EFEFEF} 
					lignes & Valeur de n1  & Valeur de n2  \\ \hline
					1&         3 &         rien     \\ \hline
					2&         3  &        5.5      \\ \hline
					3&        5.5  &       5.5       \\ \hline
					4&        5.5  &       5.5       \\ \hline
					
				\end{tabular}
			\end{center}
			\item Non
		\end{enumerate}
		
		\item 
		\begin{enumerate}[label=\alph*.]
			\item Compléter sur le tableau d'exécution ci-dessous, en donnant la valeur des variable \lstinline{n1} et \lstinline{n2} après l'exécution de chaque l'instruction numérotée.
			\begin{center}
				\begin{tabular}{|l|p{5cm}|p{5cm}|}
					\hline
					\rowcolor[HTML]{EFEFEF} 
					lignes & Valeur de n1  & Valeur de n2  \\ \hline
					1&    3      &      Rien        \\ \hline
					2&    3      &     5.5         \\ \hline
					3&    3      &     3         \\ \hline
					4&    3      &      3        \\ \hline
					
				\end{tabular}
			\end{center}
			\item Non	
		\end{enumerate}
		\item  On utilise un variable temporaire:
			\begin{lstlisting}[numbers=none]
n1 = 3
n2 = 5.5
tmp = n1
n1 = n2
n2 = tmp
			\end{lstlisting}
			ou l'affectation multiple
			\begin{lstlisting}[numbers=none]
n1, n2 = n2, n1
			\end{lstlisting}
	\end{enumerate}

\end{correction}
\finexo



\exo{}  ~\\ 
{}  ~\\ 
 On considère le programme dans la colonne Code ci-dessous.
\begin{center}
	\begin{tabular}{|l|l|p{5cm}|}
		\hline
		\rowcolor[HTML]{EFEFEF} 
		lignes & code & Valeur de x  \\ \hline
		1&          x =  0&              \\ \hline
		2&          a = 1&              \\ \hline
		3&          b = 9&              \\ \hline
		4&          c = 4&              \\ \hline
		5&          d = 8&              \\ \hline
		6&          x = a&              \\ \hline
		7&          x = x * 10 + b&              \\ \hline
		8&          x = x * 10 + c&              \\ \hline
		9&          x = x * 10 + d&              \\ \hline
	\end{tabular}
\end{center}

\begin{enumerate}[label=\alph*)]
	\item Compléter la colonne de droite en donnant la valeur courante de la variable x après que la k-ième  ligne ait été exécutée.
	\item Pour vous aider à comprendre le programme, essayer de l'exécuter dans Thonny instruction après instruction en mode débogueur (voir le tutoriel) en regardant l'évolution de la variable x dans la fenêtre \textit{variables}.
\end{enumerate}

\finexo

\newpage

\exo{}  ~\\ 
\begin{enumerate}[label=\alph*)]
	\item  	Indiquez la valeur des variables à l’issue de chaque ligne du programme suivant.
	\begin{lstlisting}[numbers=none]
		a = 4.5
		b = 5
		c = a + b
		d = c / 2
	\end{lstlisting}
	\item 	Quel est le type de chaque variable ?
	\item On considère maintenant que \lstinline{a} et \lstinline{b} correspondent à des notes. Réécrivez le programme en utilisant des noms de variables plus représentatifs (pour les 4 variables).
\end{enumerate}
\begin{correction}
	~\\ 
	
\end{correction}
\finexo


\exo{}  ~\\ 
 On considère le programme \py suivant.
\begin{lstlisting}[numbers=none]
a = 8
b = 3
a = a - 4
b = 2 * b
a = a + b
print(a)
\end{lstlisting}
\begin{enumerate}[label=\alph*)]
	\item  	Combien de variables sont utilisées ?
	\item Quelle est la valeur finale de la variable \lstinline{a} ?
	\item Vérifiez votre réponse à la question précédente en exécutant le code sur Thonny.
	\item Il est possible d’afficher plusieurs valeurs avec la fonction \lstinline{print}. Par exemple, si on veut afficher les valeurs des variables \lstinline{a} et \lstinline{b} on écrit simplement \lstinline{print(a, b)}. Modifiez la dernière ligne du programme et exécutez-le.
\end{enumerate}
\finexo

\exo{}  ~\\ 
 On considère le programme \py suivant.
\begin{lstlisting}[numbers=none]
a = 5
b = a + 1
b = b + 2
c = b - a
print(c)
\end{lstlisting}
\begin{enumerate}[label=\alph*)]
	\item  	Qu’affiche ce programme ?
	\item Vérifiez votre réponse à la question précédente en exécutant le code sur Thonny.
\end{enumerate}
\begin{correction}
	~\\ 
	\lstinputlisting{codes/ex8.py}
\end{correction}
\finexo

\exo{}  ~\\ 
 Écrire un programme qui initialise une variable appelée \lstinline{a} avec la valeur 6, puis affiche le contenu de \lstinline{a}, la divise par trois et la remet dans \lstinline{a} et finalement affiche la nouvelle valeur de \lstinline{a} à l'écran.

\finexo

\newpage
\exo{}  ~\\ 
 Est-ce que les 4 scripts (petits programmes) sont valides? Si oui que s’affichera-t-il à l’écran? Dans le cas contraire, expliquer ce qui ne va pas.  
\begin{multicols}{2}
	\begin{lstlisting}[numbers=none]
a, b = 'Mahe', 'Moeve'
print('a = ', a, ' et b = ', b)
	\end{lstlisting}
	\begin{lstlisting}[numbers=none]
		
a = b
b = 1
print('a = ', a, ' et b = ', b)
	\end{lstlisting}
	\begin{lstlisting}[numbers=none]
a, b = Mahe, Moeve
print('a = ', a, ' et b = ', b)
	\end{lstlisting}
	\begin{lstlisting}[numbers=none]
a, b = 2, 3
a = b
b = a
print('a = ', a, ' et b = ', b)
	\end{lstlisting}
\end{multicols}
\vspace{-5mm}
Exécuter ces programme avec Thonny afin de vérifier si vos réponses sont correctes.

\begin{important}
	Avant de pouvoir utiliser une variable il faut toujours la déclarer, c'est-à-dire en \py faire une affectation à la variable avec une valeur valide
\end{important}

\finexo

\exo{}  ~\\ 
 On considère le programme de calcul suivant.
\begin{itemize}
	\item \lstinline{a} prend la valeur 5
	\item  Multiplier \lstinline{a}  par 3
	\item Soustraire 4 au résultat
	\item Elever le résultat au carré
	\item Afficher le résultat
\end{itemize}
Ecriver un programme Python permettant de coder ce programme de calcul.
\begin{correction}
	~\\ 
	\lstinputlisting{codes/ex10.py}
\end{correction}
\finexo

\begin{apprendre}
	Pour demander à l’utilisateur de déterminer le contenu d’une variable, on utilise l’instruction \lstinline{input}.
	\begin{lstlisting}[numbers=none]
prenom=input("quel est ton prenom? ")
print(prenom)
	\end{lstlisting}
	Par défault la fonction \lstinline{input} retourne une chaîne de caractère. Si l'on souhaite demander à l'utilisateur un nombre, il faut entourner la fonction \lstinline{input} du type souhaité
	\begin{lstlisting}[numbers=none]
age = int(input("quel est ton age? "))        # age sera de type int
taille = int(input("quel est ta taille? "))   # taille sera de type float
print("j'ai ", age, " et je mesure ", taille)  
	\end{lstlisting}
\end{apprendre}

\exo{}  ~\\ 
 Écrire un script en Python qui demande à l'utilisateur, son prénom, son nom et son âge et qui réalise un affichage du type \lstinline{"Bonjour je m'appelle Rene Descartes, j'ai 425 ans!"}.
 \begin{correction}
 	~\\ 
 	\lstinputlisting{codes/ex11.py}
 \end{correction}
\finexo

\newpage

\exo{}  ~\\ 
  Un boulanger désire un programme qui demande à l'utilisateur le nombre de baguettes qu'il désire, qui calcule le prix total (sachant qu'une baguette coûte 2.50CHF) et qui affiche le prix que l'utilisateur doit payer. 
\begin{enumerate}[label=\alph*)]
	\item  	Testez le script suivant :
	\begin{lstlisting}[numbers=none]
nombre=input("Combien de baguettes désirez-vous ?")
prix = nombre * 1.1
print("Vous avez à payer",prix,"euros.")
	\end{lstlisting}
	\item Quel message d'erreur obtenez-vous ?
	
	\item Modifier le code ci-dessus pour que le programme fonctionne.
\end{enumerate}
\begin{correction}
	~\\ 
	\begin{enumerate}[label=\alph*)]
		\item  	\ 

		\item Un problème de type
		
		\item 
		\begin{lstlisting}[numbers=none]
nombre=float(input("Combien de baguettes désirez-vous ?"))
prix = nombre * 1.1
print("Vous avez à payer",prix,"euros.")
		\end{lstlisting}
	\end{enumerate}
\end{correction}
\finexo

\exo{}  ~\\ 
  L'Indice de Masse Corporelle (IMC) est un indicateur chiffré utilisé en médecine. L'IMC d'une personne est donné par la formule $IMC = \frac{masse}{taille ^2}$  où la masse est en kilos et la taille en mètres. 

Proposer un algorithme qui demande à l'utilisateur sa taille et sa masse puis qui affiche l'IMC de la personne. 
\begin{correction}
	~\\ 
	\lstinputlisting{codes/ex13.py}
\end{correction}
\finexo

\exo{}[Apprendre en  autonomie avec france IOI]  ~\\ 
Allez sur le site de france IOI sur \url{http://www.france-ioi.org/}, puis valider la séquence 3 "\textit{Calculs et découverte des variables}".
\finexo

% Solution		
		\newpage
		\setcounter{page}{1}
		\setcounter{section}{\numero}
		\Closesolutionfile{mycor}
		\titleformat{\section}[hang]{\Large \bfseries}{Corrigé Série \thesection:\ }{0pt}{}
		

		\fancyhead[CO,CE]{\sc Corrigé Série \arabic{section} \hspace{0.5mm}}
		\section{}
		\Readsolutionfile{mycor}
	\end{document}