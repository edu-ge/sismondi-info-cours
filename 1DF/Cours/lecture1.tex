\mainsection{1}{Lecture Title}{dd/mm/yyyy}

\section{Title 1}

\subsection{Title 1.1}
	You can use the custom putfigure command to add figures to the document. the command takes 6 arguments - how much horizontal size to create a bounding box, width, height, path to image, caption, label.
	
\verb!\putfigure{1.0}{0.7}{0.3}{Images/rl1}{Many Faces of RL}{fig1_1}! \\

\putfigure{1.0}{0.7}{0.3}{Images/rl1}{Many Faces of RL}{fig1_1}

\subsection{Title 1.2}

\section{Title 2}
	Adding multiple images using putfigure with minipage \\
\verb!\begin{minipage}{\textwidth}! \\
\verb!    \centering! \\
\verb!        \putfigure{0.49}{1.0}{0.2}{Images/rl2}{A generic RL schematic diagram}{fig1_2}! \\
\verb!        \putfigure{0.49}{1.0}{0.2}{Images/rl3}{An RL agent with environment loop}{fig1_3}! \\
\verb!\end{minipage}! \\

\begin{minipage}{\textwidth}
	\centering
		\putfigure{0.49}{1.0}{0.2}{Images/rl2}{A generic RL schematic diagram}{fig1_2}
		\putfigure{0.49}{1.0}{0.2}{Images/rl3}{An RL agent with environment loop}{fig1_3}
\end{minipage}

	You can use the myequations command to add any equation to the list of equations in contents. Just pass the name for that equation with the command \\


\verb!\begin{align}! \\
\verb!    \S_t = \f(\H_t)! \\
\verb!\end{align}! \\
\verb!\myequations{Generic state RL system}! \\

\begin{align}
	\S_t = \f(\H_t)	
\end{align}
\myequations{Generic state RL system}